\documentclass{beamer}
\usetheme{Warsaw}

\usepackage{color}
\usepackage{CJK}
\usepackage{listings}
\usepackage{url}
\usepackage{booktabs}
\usepackage{pgf}
\usepackage{multicol}
%\usepackage{algorithm}
%\usepackage{algorithmic}

% to specify the size of fonts in lst environment

\lstset{showstringspaces=false}
\lstset{language=C}
\lstset{basicstyle=\tt}
\lstset{keywordstyle=\color{blue}}
\lstset{emphstyle=\color{red}}
\lstset{numbers=left}
\lstset{frame=single}
\lstset{rangeprefix=\/*\ }
\lstset{rangesuffix=\ *\/}
\lstset{includerangemarker=false}

%% Program listing macro for the first page with default setting.  

\newcommand{\programlisting}[2]
{\renewcommand{\lstlistingname}{Example}
\lstinputlisting[caption={#2}]{#1.c}}

\newcommand{\programlistingoption}[3]
{\renewcommand{\lstlistingname}{Example}
\lstinputlisting[caption={(#1.c) #2},#3]{#1.c}}

\newcommand{\executionlisting}[2]
{{\renewcommand{\lstlistingname}{Execution}
\lstinputlisting[caption={}]{#1.out}}}

\newcommand{\executionlistingoption}[3]
{{\renewcommand{\lstlistingname}{Execution}
\lstinputlisting[caption={},#3]{#1.out}}}

\newcommand{\filelisting}[2]
{\renewcommand{\lstlistingname}{Example}
\lstinputlisting[caption={#2}]{#1}}

\newcommand{\prototype}[2]
{{\renewcommand{\lstlistingname}{Prototype}
\lstinputlisting[caption={#2}]{#1.h}}}

\newcommand{\pragma}[2]
{{\renewcommand{\lstlistingname}{Pragma}
\lstinputlisting[caption={#2}]{#1.h}}}


\begin{document}
\begin{CJK}{UTF8}{bsmi}

\title{Parallel Algorithm Evaluation}

\author{Pangfeng Liu \\ National Taiwan University}

\begin{frame}
\titlepage
\end{frame}

\section{Introduction} 

\begin{frame}
\frametitle{Introduction}
\begin{itemize}
\item Algorithm describes the procedure that solves a problem.
\item ``Algorithmic thinking'' is very important when one wants to use
  computer to solve problems; in a sense that computers can easily
  realize an algorithm with its powerful computation capability.
\item This is not confined to solving problem with computers, but in
  this lecture we focus on algorithm for computer.
\end{itemize}
\end{frame}

\begin{frame}
\frametitle{Computers}
\begin{itemize}
\item Some algorithms are particularly suitable for computers.
\item Consider Sudoku problem.  You want to place the number of 1 to 9
  into a 9 by 9 matrix so that every row, every column and the nine 3
  by 3 sub-matrices have numbers from 1 to 9.
\item We can use a ``trial-and-error'' algorithm to solve this
  problem.  However, this is time consuming and error-prone for human
  to execute this algorithm.
\item In contrast a computer programmer can easily convert the
  ``trial-and-error'' algorithm to a computer program that solves this
  problem in no time.
\end{itemize}
\end{frame}

\begin{frame}
\frametitle{Definition}
\begin{itemize}
\item An algorithm must describe the {\em detailed operations} one
  wants to perform.
\item These operations must be {\em well-defined} within the underling
  model (more on this later).
\item The algorithm must also describe the {\em temporal dependency}
  of these operations, i.e., loop, synchronization, etc.
\end{itemize}
\end{frame}

\begin{frame}
\frametitle{Example}
\begin{itemize}
\item How to pick the largest number from a set of numbers.
\item Look through all the numbers one at a time.
\item Compare a number with the current largest one you have seen.
\item If the number is the larger, replace the current largest one
  with it.
\item Finally you have the largest number.
\end{itemize}
\end{frame}

\begin{frame}
\frametitle{Example}
\begin{itemize}
\item The operations, i.e., examine, compare, and replace, are well
  defined.
\item The temporal sequences of these operations are also well defined
  -- ``look through all'', ``if something happens then does this''.
\item This is an algorithm indeed.
\end{itemize}
\end{frame}

\begin{frame}
\frametitle{Discussion}
\begin{itemize}
\item Give an example of algorithm.
\end{itemize}
\end{frame}

\begin{frame}
\frametitle{Parallel Algorithms}
\begin{itemize}
\item Now we extend the concept to parallel computation.
\item A {\em sequential} algorithm describes the procedure that solves
  a problem with a computer.
\item A {\em parallel} algorithm describes the procedure that solves a
  problem with a parallel computer.
\item In this course we will focus on parallel algorithms.
\end{itemize}
\end{frame}

\begin{frame}
\frametitle{Parallel Algorithms}
\begin{itemize}
\item A sequential algorithm is relatively easy to describe.
\begin{itemize}
\item The timing sequence is for one processor only.
\end{itemize}
\item A parallel algorithm is {\em harder} to describe since we have
  to deal with multiple entities working concurrently.
\begin{itemize}
\item The timing constraints are about multiple processors, hence much
  difficult to describe and analyze.
\item Different processors may need to access the same data, and may
  have race condition.
\end{itemize}
\end{itemize}
\end{frame}

\begin{frame}
\frametitle{Model}
\begin{itemize}
\item Any algorithm has an underlining assumption on what can be done,
  for example, a sorting algorithm assumes that keys can be compared
  in $O(1)$, i.e., a constant amount of time.
\item We then follow these assumptions to estimate the {\em cost} of
  the algorithm we are considering.
\item The purpose of the model is to estimate the cost accurately, so
  it has to be realistic, which means it must resemble the real
  hardware to be meaningful.
\end{itemize}
\end{frame}

\begin{frame}
\frametitle{Analysis}
\begin{itemize}
\item The process of estimating the cost of an algorithm is {\em
  algorithm complexity analysis}.
\item Note that we are not actually estimate the running time of the
  algorithm since this is a moving target.
\item Instead we measure {\em the number of times} certain operations
  (e.g., computation or communication), and use them as the estimate
  on the cost of the algorithm.
\item We want to design algorithms will low costs.
\end{itemize}
\end{frame}


\begin{frame}
\frametitle{Analysis}
\begin{itemize}
\item A sequential algorithm is usually easy to analyze.
\item A parallel algorithm is much more difficult to analyze since
  different models, i.e., the assumption on how the parallel computer
  can do, have different algorithmic characteristics.
\end{itemize}
\end{frame}

\begin{frame}
\frametitle{Analysis}
\begin{itemize}
\item A shared memory model algorithm on a multiprocessor can be very
  different from a distributed memory algorithm on a multicomputer.
\item Nevertheless there are certain criteria that we may follow,
  mostly from what the actual CPU can do in a fixed amount of time.
\item In this lecture I will try my best to focus on the parallel
  computing issues, instead of the computation models, i.e., I want to
  have generic analysis, instead of model specific analysis.
\end{itemize}
\end{frame}

\begin{frame}
\frametitle{Discussion}
\begin{itemize}
\item Give an example of algorithm analysis, like doing an barrier
  synchronization on a distributed memory parallel computer..
\end{itemize}
\end{frame}

\section{Evaluation}

\subsection{Speedup}

\begin{frame}
\frametitle{Speedup}
\begin{itemize}
\item How to determine which parallel algorithm is good, and which is
  bad?
\item We use {\em speedup} as a metric to evaluate parallel
  algorithms, which is the ratio between the best sequential time $T_s$
  and the parallel time $T_p$.
\begin{equation} 
k = {{T_s}\over{T_p}}
\end{equation}
\item Note that we need to use the {\em best} $T_s$ for a meaningful
  comparison.
\end{itemize}
\end{frame}

\begin{frame}
\frametitle{Speedup Improvement}
\begin{itemize}
\item Remember that performance is paramount for parallel processing
  (remember the racecar?), so speedup is essential.
\item There are two ways to improve speedup.
\begin{itemize}
\item The {\em right} way is to reduce the parallel time.
\item The {\em wrong} way is to increase the sequential time.  That is why
  we need to use the best sequential algorithm.
\end{itemize}
\end{itemize}
\end{frame}

\begin{frame}
\frametitle{Lesson} \huge When you hear people talking about speedup,
always make sure that you know the definition of their ``speedup''.
\end{frame}

\begin{frame}
\frametitle{Banana and Orange}
\begin{itemize}
\item In my opinion, we should always calculate the speedup with {\em
  the same basis}.
\item What do we get if we compare the sequential time of a sequential
  program running on a CPU, with the parallel time of the same program
  running on five GPU's, and get the speedup of 136?
\end{itemize}
\end{frame}

\begin{frame}
\frametitle{Banana and Orange}
\begin{itemize}
\item The comparison between one CPU ad five GPU's is not {\em
  quantitatively} meaningful because we cannot derive any {\em
  quantitative} conclusion on how well we are doing.
\item We may have a terrible implementation and still get good speedup
  because the CPU is running slowly, or the GPU's are running fast
  like hell.
\item You are comparing banana and orange.
\end{itemize}
\end{frame}

\begin{frame}
\frametitle{Relative Speedup}
\begin{itemize}
\item A speedup comparison is quantitatively meaningful if we can relate
  the speedup with the extra amount of resources we use in the
  parallel computation.
\item Another speedup metric is to compare the parallel time of using
  $k$ processors with the {\em parallel time} of using a single a
  single processor.  This is usually referred to as {\em relative
    speedup}.
\item As the definition points out, a relative speedup is how well we
  parallelize a computation.
\end{itemize}
\end{frame}


\begin{frame}
\frametitle{Discussion}
\begin{itemize}
\item Describe the concept of speedup by examples.
\end{itemize}
\end{frame}

\begin{frame}
\frametitle{Overheads}
\begin{itemize}
\item Note that the execution time of a sequential algorithm may be
  smaller than a parallel program using a single processor.
\item There are inherent overheads in the execution of parallel
  program, even if you are using only one processor.
\end{itemize}
\end{frame}

\begin{frame}
\frametitle{Overheads}
\begin{itemize}
\item A parallel system may need to start up.
\item The parallel program may use extra parallel library, which may
  incur extra overhead.
\item There may be synchronization overhead, even if only one
  processor is involved.
\item To keep the following theoretic discussion (on efficiency and
  work) simple we will assume that these two are the same, but keep in
  mind that there are always overheads in parallel programming.
\end{itemize}
\end{frame}


\begin{frame}
\frametitle{Parallelism}
\begin{itemize}
\item The speedup $k$ alone is not sufficient to evaluate the
  algorithm, since we do not know how many {\em processors} are used.
\item Let $p$ be the number of processors used in the parallel
  algorithm.
\item The speedup $k$ is between $0$ and $p$.
\begin{equation}
0 < k \leq p
\end{equation}
\item If the speedup is close to $p$ then we have a {\em linear
  speedup}.
\end{itemize}
\end{frame}

\begin{frame}
\frametitle{Theoretical Bounds} We can argue mathematically that
speedup $k$ is between $1$ and $p$.
\begin{equation}
1 \leq k \leq p
\end{equation}
\end{frame}


\begin{frame}
\frametitle{Proof}
\begin{itemize}
\item We can use one processor to simulate the sequential algorithm,
  hence $1 \leq k$.
  \begin{itemize}
  \item Recall that we assume that we do not have overheads due to
    parallelization.
  \end{itemize}
\end{itemize}
\end{frame}

\begin{frame}
\frametitle{Proof}
\begin{itemize}
\item We can simulate one step of the parallel algorithm with $p$
  steps on one processor.
\begin{itemize}
  \item If that takes less than $k$ steps, we have an algorithm that
    runs faster than our optimal sequential algorithm, which is a
    contradiction.
  \item Note that we need to assume that the sequential algorithm is
    optimal.
\end{itemize}
\end{itemize}
\end{frame}


\begin{frame}
\frametitle{In Practice}
\begin{itemize}
\item In rare occasion we may have $T_s < T_p$.  This is usually
  caused by a very small workload and a tremendous amount of overheads in
  parallel execution.
\item Recall that we have all the overheads due parallelization.  If
  the benefits of parallelization is not enough to compensate the
  overheads, we do not have any speedup.
\item Some problems are better left alone (e.g., inherently sequential
  problems).
\end{itemize}
\end{frame}

\begin{frame}
\frametitle{In Practice}
\begin{itemize}
\item Also in rare occasion we may even have $k > p$, which is {\em
  super-linear speedup}.  
\item This is usually due to the size of the working set of the
  program.
\end{itemize}
\end{frame}

\begin{frame}
\frametitle{Caching Effects}
\begin{itemize}
\item If a problem has a very large working set, then it is impossible
  to fit it into the cache of a single computer.  As a result frequent
  cache misses degrades performance significantly.
\item If we divide the problem into many small sub-problems. and then
  run them in parallel, then it is likely that each working set of
  these small sub-problems will fit into the cache of a processor,
  hence having amazing speedup.
\end{itemize}
\end{frame}

\begin{frame}
\frametitle{Lesson} \huge Always question theory in the context of
``real world''.
\end{frame}

\begin{frame}
\frametitle{Discussion}
\begin{itemize}
\item Describe the relation of speedup and the number of processors.
\end{itemize}
\end{frame}

\subsection{Efficiency}

\begin{frame}
\frametitle{Efficiency} Another metric to evaluate parallel algorithms
is {\em efficiency}, which is defined as the speedup divided by the
number of processors.
\begin{equation}
e = {k \over p} =  {T_s \over {p T_p}}
\end{equation}
\end{frame}

\begin{frame}
\frametitle{Efficiency}
\begin{itemize}
\item It is easy to see that efficiency $e$ is between 0 and 100\%, if
  the speed up is between 0 and $p$, where $p$ is the number of
  processors.
\item If we fully parallelize a computation, the efficiency is 100\%.
\item One can think of the efficiency as ``how well we parallelize the
  computation per processor?'' -- that is, it is the CP value, or
  performance/cost ratio.
\end{itemize}
\end{frame}

\begin{frame}
\frametitle{Discussion}
\begin{itemize}
\item Describe the relation of efficiency and speedup.
\end{itemize}
\end{frame}

\subsection{Work}

\begin{frame}
\frametitle{Work} 

Another metric for parallel algorithm evaluation is {\em work}, which
is the product of the number of the processor and the parallel time.

\begin{equation}
w_p = p T_p
\end{equation}
\end{frame}

\begin{frame}
\frametitle{Energy}
\begin{itemize}
\item The concept of {\em work} is like man-month; you used this many
  processors for this period of time.
\item It is like that the energy is the product of power and time; you
  used this much power for a period of time, then you use this much
  energy.
\end{itemize}
\end{frame}

\begin{frame}
\frametitle{Extra Work}
\begin{itemize}
\item The work done by a sequential algorithm is $w_s = T_s$, and the
  work done by a parallel algorithm is $w_p = p T_p$.
\item We assume that all the work done by the sequential algorithm is
  {\em necessary}; that means all the work done by the sequential
  algorithm is {\em essential}.
\item Now if the work done by the parallel algorithm is larger than
  the work done by the sequential algorithm, then the difference is {\em
    non-essential} in solving this problem.
\item This is what we called {\em overheads}.
\end{itemize}
\end{frame}


\begin{frame}
\frametitle{Extra Work}
\begin{itemize}
\item If we parallel a computation well, the two works done by the
  sequential algorithm and the parallel algorithm should be similar,
  which implies three things.
\begin{itemize}
\item The Efficiency $e$ should be close to 1.
\item The speedup $k$ should be close to $p$.
\item $w_s$ should be close to $w_p$.
\end{itemize}
\end{itemize}

\begin{equation}
w_p = p T_p = p {T_s \over k} = {w_s \over e}
\end{equation}
\end{frame}

\begin{frame}
\frametitle{Discussion} Compute the speedup, efficiency, work for the
following, and describe the circumstance each of these action is
suitable for
\begin{itemize}
\item Using one processor and the time is 40 minutes.
\item Using two processors and the time is 24 minutes.
\item Using four processors and the time is 16 minutes.
\end{itemize}
\end{frame}

\section{Theory}

\subsection{Amdahl's Law}

\begin{frame}
\frametitle{Amdahl's Law} 
\begin{itemize}
\item The speedup of a program using multiple processors in parallel
  computing is limited by the time needed for the sequential fraction
  of the program.
\end{itemize}
\end{frame}

\begin{frame}
\frametitle{Amdahl's law} 
\begin{itemize}
\item Let the portion of inherent sequential in a computation be $x$,
  the part that can be parallelized will be $1 - x$.
\item Assume that the sequential time is $1$, then the parallel time
  will be at least $x + {{1-x} \over p}$ while using $p$ processors.
\end{itemize}

\begin{equation}
k = {1 \over {x + {{1-x} \over p}}} \leq {1 \over x}
\end{equation}
\end{frame}

\begin{frame}
\frametitle{Amdahl's Law} 
\begin{itemize}
\item Amdahl's law says that if you have 20\% of you code is
  inherently sequential, the speedup could not be more than $5$.
\item That hurts!
\end{itemize}
\end{frame}

\begin{frame}
\frametitle{Amdahl's Law} 
\centerline{\pgfimage[width=0.6\textwidth]{AmdahlsLaw}}
\footnote{\url{http://upload.wikimedia.org/wikipedia/commons/6/6b/AmdahlsLaw.png}}
\end{frame}

\begin{frame}
\frametitle{Implications}
\begin{itemize}
\item Every program has an {\em inherently} sequential part, which
  limits the speedup.
\item If the inherent part is large, then the program is inherently
  sequential and we do not want to parallelize it; we have to admit
  that there are computations that cannot be parallelized efficiently.
\item The lesson here is to recognize the cases that have only limited
  parallelism.
\end{itemize}
\end{frame}

\begin{frame}
\frametitle{Discussion}
\begin{itemize}
\item Describe Amdahl's Law.
\end{itemize}
\end{frame}

\subsection{Gustafson's Law} 

\begin{frame}
\frametitle{Two Ways}
\begin{itemize}
\item Remember there are two ways to improve speedup.  We either
  increase the sequential time, or we decrease the parallel time.
\item In many cases we can increase the sequential time by increasing
  the {\em problem size}.  If the impact of the increased problem size
  on sequential time is larger than on the parallel time, then we have
  a ``better'' speedup.
\end{itemize}
\end{frame}

\begin{frame}
\frametitle{Gustafson's Law} 
\begin{itemize}
\item Computations involving {\em arbitrarily} large data sets can be
  efficiently parallelized.
\item This is true only when the ``inherently sequential'' part is a
  constant, i.e., it is not a function of the problem size.
\end{itemize}
\end{frame}

\begin{frame}
\frametitle{Gustafson's Law} 
\begin{itemize}
\item Let the portion of inherent sequential in a computation be $x$,
  the part that can be parallelized will be $1 - x$.
\item Instead of a fixed problem size, we increase the number of
  processors to $p$.  Then after $x$ of inherent sequential work, $p$
  processors does $p (1 - x)$ amount of work.
\end{itemize}
\end{frame}

\begin{frame}
\frametitle{Gustafson's Law} 
\begin{itemize}
\item The sequential time is now at least $x + p (1-x)$, since the
  amount of work done by $p$ processors is $p (1 - x)$, and the
  sequential program only has one processor to do it.
\item The parallel time is $x + (1-x) = 1$ with $p$ processors.
\item We can have speedup close to $p$ for any $x$ by increasing $p$.
\end{itemize}
\begin{equation}
k = p (1 - x) = p - px
\end{equation}
\end{frame}

\begin{frame}
\frametitle{Gustafson's Law} 
\centerline{\pgfimage[width=0.7\textwidth]{Gustafson}}
\footnote{\url{http://slashdot.org/topic/wp-content/uploads/2012/08/Screen-Shot-2012-08-29-at-12.19.53-PM-618x425.png}}
\end{frame}

\begin{frame}
\frametitle{Notes}
\begin{itemize}
\item We assume that $x$ portion of sequential part is still
  sufficient even if we increase the work $p$ times.
\item In many cases this is true if $x$ only involves setting up the
  systems, and data can be read in parallel.
\item If data cannot be read in parallel, then $x$ will be a function
  of the problem size, which invalidates the analysis.
\end{itemize}
\end{frame}

\begin{frame}
\frametitle{Notes}
\begin{itemize}
\item Despite the possible complications, this observation is generally
  useful because $x$ is usually a small constant if only system setup
  is involved.  In addition, most computation has complexity like
  $O(n^2)$ so the work of reading the data, even if done sequentially,
  is not comparable to the actual computation.
\end{itemize}
\end{frame}

\begin{frame}
\frametitle{Notes}
\begin{itemize}
\item If we really really want to have a clear picture about speedup,
  we would also like to know the problem size when the speedup is
  measured.
\item If we have a {\em good} speedup even if problem size is small,
  then we really have a good parallel implementation.  Otherwise it
  could be a fabricated phenomenon with the use (or abuse) of
  Gustafson's Law.
\item When you examine a speedup report, always question the
  comparison basis and the problem size.
\end{itemize}
\end{frame}

\begin{frame}
\frametitle{Discussion}
\begin{itemize}
\item Describe Gustafson's Law.
\end{itemize}
\end{frame}

\end{CJK}
\end{document}

