\documentclass{beamer}
\usetheme{Warsaw}

\usepackage{color}
\usepackage{CJK}
\usepackage{listings}
\usepackage{url}
\usepackage{booktabs}
\usepackage{pgf}
\usepackage{multicol}
%% \usepackage{algorithm}
%% \usepackage{algorithmic}

\input{../slide-macro}

\begin{document}
\begin{CJK}{UTF8}{bsmi}

\title{Parallel Algorithm Principles}

\author{Pangfeng Liu \\ National Taiwan University}

\begin{frame}
\titlepage
\end{frame}

\begin{frame}
\frametitle{Principles} There are three basic principles in improving
the efficiency of parallel computing.
\begin{itemize}
\item Even partition
\item Communication reduction
\item Efficient Implementation
\end{itemize}
\end{frame}

\section{Even Partition}

\begin{frame}
\frametitle{Partition}
\begin{itemize}
\item Partition is an essential parallel algorithm design technique.
\item As in a sequential divide-and-conquer algorithm, the problem is
  first partitioned (divided) into sub-problems.
\item Unlike a sequential divide-and-conquer algorithm, a parallel
  algorithm solves (conquers) the sub-problem {\em in parallel}.
\item Some communication may be necessary since the sub-problems may depend on other tasks or may need to transfer data among
  themselves.
\item Finally, we combine the answers from individual sub-problems into the final answer.
\end{itemize}
\end{frame}

\begin{frame}
\frametitle{Partition}
\begin{itemize}
\item Partition is the first step in a divide-and-conquer algorithm.
\item One can partition the data and the process {\em data partitioning}.
\item Or one can partition the main loop of the computation, and it is {\em loop} partitioning.
\item The partition has a significant impact on the overall performance.
\end{itemize}
\end{frame}

\begin{frame}
\frametitle{Discussion}
\begin{itemize}
\item Give an example of divide-and-conquer computation.
\end{itemize}
\end{frame}

\begin{frame}
\frametitle{Partition Principles}
There are two important issues in partitioning.
\begin{itemize}
\item Even workload distribution
\item Proper granularity
\end{itemize}
\end{frame}

\subsection{Even Workload Distribution}

\begin{frame}
\frametitle{Even Workload Distribution}
\begin{itemize}
\item We want to distribute the workload among processors to minimize the maximum workload among all processors.
\item The execution time of a parallel program is the execution time of the {\em slowest} processor involved, which is usually the processor that has the maximum workload.
\begin{itemize}
\item This is the {\em makespan} of the execution time.  
Note that usually, we are interested in the makespan of the execution, not the sum or the average of the execution time of processors.
\end{itemize}
\end{itemize}
\end{frame}

\begin{frame}
\frametitle{Idle v.s. Busy}
\begin{itemize}
\item Uneven distribution of workload leaves some processor idle while others are busy.
\item If everyone is busy all the time, then the workload is evenly distributed.
\end{itemize}
\end{frame}

\begin{frame}
\frametitle{Workload Estimation}
\begin{itemize}
\item To distribute the workload evenly, one needs to predict the workload accurately.
\item For data-parallel computation, one can associate the computation with the data.
If we further assume that the computation workload on every data is about {\em the same}, then we can estimate the workload by counting the {\em number} of data each processor is assigned.
\end{itemize}
\end{frame}

\begin{frame}
\frametitle{Workload Estimation}
\begin{itemize}
\item For task parallel computation, we must predict the workload of sub-problems.
\item It is difficult to estimate the workload of tasks, so profiling or programmer intervention is necessary.
\end{itemize}
\end{frame}

\begin{frame}
\frametitle{Discussion}
\begin{itemize}
\item Give an example to illustrate the importance of even workload distribution.
\end{itemize}
\end{frame}

\subsection{Proper Granularity}

\begin{frame}
\frametitle{Granularity}
\begin{itemize}
\item The granularity is the basic unit in partitioning.
\item For data-parallel computation, it indicates the smallest chunk of data while assigning data chunks to processors.
\item For task-parallel computation, it indicates the smallest chunk of task while assigning tasks to processors.
\begin{itemize}
\item Recall that we can always {\em refine} a step of our algorithm into finer steps.
\end{itemize}
\end{itemize}
\end{frame}

\begin{frame}
\frametitle{The Size}
\begin{itemize}
\item It is always easier to balance the workload if the granularity is {\em small} because it is always easier to distribute a set of objects evenly if we can {\em cut them into small pieces}.
\item There will be much more overhead in assigning these chunks to processors because the mapping table will be larger and in scheduling and synchronizing the processors because the number of these operations will increase.  
\item More details on the communication later.
\end{itemize}
\end{frame}

\begin{frame}
\frametitle{Fine and Coarse}
\begin{itemize}
\item {\em Fine grain} parallelism partitions data/task into very small pieces, then assigns them to processors for processing.
\begin{itemize}
\item Suitable for system that can spawn a large number of threads with low cost, e.g., GPU.
\end{itemize}
\item {\em Coarse grain} parallelism partitions data/task into very large pieces, then assigns them to processors for processing.
\begin{itemize}
\item Suitable for system that can only spawn a limited number of threads, and the thread creation is expensive, e.g., CPU.
\end{itemize}
\end{itemize}
\end{frame}

\begin{frame}
\frametitle{Discussion}
\begin{itemize}
\item Give an example to illustrate the importance of granularity in
  partitioning workload.
\end{itemize}
\end{frame}

\section{Communication Optimization}

\begin{frame}
\frametitle{Communication Reduction}
\begin{itemize}
\item Communication is inevitable because multiple processors are
  working on the {\em same} problem.
\item Communication is overhead -- it does not appear in a sequential
  computation.
\item Communication should be reduced.
\end{itemize}
\end{frame}

\begin{frame}
\frametitle{Principles}
\begin{itemize}
\item There are two basic principles to reduce communication.
\begin{itemize}
\item Low synchronization overheads
\item Data locality
\end{itemize}
\end{itemize}
\end{frame}

\subsection{Synchronization}

\begin{frame}
\frametitle{Synchronization}
\begin{itemize}
\item The synchronization is inevitable in parallel and distributed
  computing because we want to coordinate the processors.
\begin{itemize}
\item Barrier synchronization
\item Before/after synchronization
\item Access synchronization
\end{itemize}
\end{itemize}
\end{frame}


\begin{frame}
\frametitle{Barrier Synchronization}
\begin{itemize}
\item A computation may proceed in {\em stages} -- all processors
  needs to finish a stage before going to the next stage.
\begin{itemize}
\item This is usually called a {\em barrier} synchronization.  For
  example, all processors must combine their partial answer into the
  final answer.
\item This usually involves {\em all} processors.
\end{itemize}
\end{itemize}
\end{frame}

\begin{frame}
\frametitle{Before/After Synchronization}
\begin{itemize}
\item In task parallelism one computation may need to precede another.
\begin{itemize}
\item You need to cook dinner before you can eat it.  This may be
  referred to as {\em before/after} synchronization.
\item This usually involves two processors -- one processor finishes a
  computation, then notifies the other processor to proceed.
\end{itemize}
\end{itemize}
\end{frame}

\begin{frame}
\frametitle{Access Synchronization}
\begin{itemize}
\item Many processor may need to access a shared variable in a shared
  memory multiprocessor.
\begin{itemize}
\item Not an issue for distributed memory multicomputer since the
  computers do not share memory.
\end{itemize}
\item If the memory access is not synchronized properly, race
  condition may occur.
\end{itemize}
\end{frame}


\begin{frame}
\frametitle{Synchronization Mechanism}
\begin{itemize}
\item Many parallel programming environments provide a mechanism for
  program to specify synchronization {\em explicitly}.
\item The synchronization should be efficient.
\item The synchronization should be scalable, i.e., it should be
  efficient even if the number of processors involved is large.
\end{itemize}
\end{frame}

\begin{frame}
\frametitle{Discussion}
\begin{itemize}
\item Give an example for each synchronization described earlier.
\end{itemize}
\end{frame}

\begin{frame}
\frametitle{Synchronization Mechanism}
\begin{itemize}
\item One can use message passing or shared memory to implement
  barrier synchronization within the same computer.
\item One can use signal inter-process communication to implement
  before/after synchronization within the same computer. 
\item One can use busy waiting or semaphore to implement the critical
  section for accessing shared variables.
\item If processors of different computers need to synchronize, they need to use a network protocol.
\end{itemize}
\end{frame}


\begin{frame}
\frametitle{Synchronization Optimization}
\begin{itemize}
\item We should reduce the number of stages.
\item The synchronization should be efficient.
\item We should carefully choose the granularity to balance synchronization and workload distribution overhead.
\begin{itemize}
\item A fine-grain parallel computation is hard to synchronize but easy to have an even workload.
\item A coarse-grain parallel computation is easy to synchronize but hard to have an even workload.
\end{itemize}
\end{itemize}
\end{frame}

\begin{frame}
\frametitle{Discussion}
\begin{itemize}
\item Describe the inter-process communication (IPC) mechanism that
  you are aware of.
\end{itemize}
\end{frame}


\subsection{Data Locality}

\begin{frame}
\frametitle{Data Locality}
\begin{itemize}
\item {\em Locality} is a trend for a program to access
  data/instruction in {\em proximity}.
\item When a program access a data/instruction, it is very likely it
  will access the same data/instruction in the dear future, or it will
  access the data/instruction nearby in the near future.
\item Computer architecture explores locality for performance.
\end{itemize}
\end{frame}

\begin{frame}
\frametitle{Temporal Locality} 
\begin{itemize}
\item When a program access a data/instruction, it is very likely to access the same data/instruction shortly.
\item If we {\em cache} this data/instruction in fast storage, then it is very likely we will be able to access the data fast.
\item Data/instruction are cached in data/instruction cache for performance.
\item CPU first tries to get the data from the cache.
\item If the CPU finds it in the cache, it uses the data; otherwise, it gets it from memory.
\item There could be several levels of caches.
\end{itemize}
\end{frame}

\begin{frame}
\frametitle{Performance}
\begin{itemize}
\item The performance comes from the difference in accessing speed to memory and cache and the probability of finding the data/cache in the cache.
\item If we can find the data/instruction in cache with high probability, i.e., with a high cache hit rate; then the performance will improve.
\item If the temporal locality is good, which means the CPU will use the same data/instruction again shortly, then we have good performance.
\end{itemize}
\end{frame}

\begin{frame}
\frametitle{Shortly?}  
What do we mean {\em shortly}?
\begin{itemize}
\item The capacity of cache is minimal.
\item When we access a data/cache, we have to place it into the cache for possible later references.
\item If the cache is {\em full}, then we have to remove some data/instructions to make space for the incoming ones.
\item {\em Shortly} means when we want to access the data/instruction we placed into cache {\em again}, it will still be there, i.e., before we remove it to make room for other data.
\end{itemize}
\end{frame}

\begin{frame}
\frametitle{Other Applications}
\begin{itemize}
\item Hard disks maintain a small cache for data stored in the disk.
\item Operating system maintains disk cache for frequently accessed
  data on disk.
\item A translation lookaside buffer (TLB) is a cache for frequently
  accesses items in the page table.
\end{itemize}
\end{frame}

\begin{frame}
\frametitle{Discussion}
\begin{itemize}
\item Give an example of temporal locality.
\end{itemize}
\end{frame}

\begin{frame}
\frametitle{Spacial Locality}
\begin{itemize}
\item When a program access a data/instruction, it will access the
  data/instruction {\em nearby} in the near future.
\item If we {\em cache} the near by data/instruction in a fast
  storage, then it is very likely we will be able to access the nearby
  data/instruction fast.
\item Parallel processing focuses on {\em spacial data locality}.
\end{itemize}
\end{frame}

\begin{frame}
\frametitle{Cache Line}
\begin{itemize}
\item Modern computer architecture does not cache data individually; instead, it caches data/instruction in the unit of a cache line.
\item A cache line consists of consecutive data/instruction in memory.
\item A cache line automatically caches nearby data/instructions and improves spatial locality.
\item Parallel programmers preserve data locality in a much higher {\em data level} when partitioning the data into chunks for processing.
\end{itemize}
\end{frame}

\begin{frame}
\frametitle{Data Level Locality}
\begin{itemize}
\item When we assign data to processors for processing, we not only
  want to distribute them evenly, we also want to preserve {\em
    spacial data locality}.
\item That means when we want to process a data, the {\em required}
  data is {\em nearby}.
\begin{itemize}
\item What is required data?
\item What is ``near by''?
\end{itemize}
\end{itemize}
\end{frame}

\begin{frame}
\frametitle{Required Data}
\begin{itemize}
\item When we process a data, we usually need {\em other} data.
\item For example, when we want to compute vector $C$, which is the
  sum of two vectors $A$ and $B$.
\item We need $A_i$ and $B_i$ to compute $C_i$, then $A_i$ and $B_i$
  are required data of $C_i$.
\end{itemize}
\end{frame}

\begin{frame}
\frametitle{Owner}
\begin{itemize}
\item We usually follow a {\em owner computes} rule.
\item If a processor is the owner of data, then it is responsible for the computation of this data.
\item The rule is simple.
\item On rare occasion we will not follow the {\em owner computes} rule.
\end{itemize}
\end{frame}

\begin{frame}
\frametitle{Placement}
\begin{itemize}
\item If the length of the vector is 32, and we have two processors, how do we assign data to processors?
\item Intuitively, we can place the first 16 elements of $A$, $B$, and $C$ to one processor and the rest to the other.
\item The workload of computing $C$ is even because each processor computes 16 elements for $C$.
\item When a processor computes a $A_i$, it can get all the required data within its memory.
\end{itemize}
\end{frame}

\begin{frame}
\frametitle{Wrong Placement}
\begin{itemize}
\item If the length of the vector is 32, and we have two processors.
\item We place the first 16 elements of $A$, $B$, and the last 16 elements of $C$ to one processor and the rest to the other.
\item The workload of computing $C$ is even because each processor will compute 16 elements for $C$.
\item When a processor computes a $A_i$, it {\em cannot} get any required data within its memory.
\item Is this good?
\end{itemize}
\end{frame}

\begin{frame}
\frametitle{Nearby}
\begin{itemize}
\item {\em Nearby} means in the same processor.
\item We can access the required data within the processor of the same
  processor by {\em memory bandwidth}.
\item We can only access the required data within the processor of other processors by {\em network bandwidth}.
\item Memory bandwidth is {\em much much larger} than network
  bandwidth.
\end{itemize}
\end{frame}

\begin{frame}
\frametitle{Local v.s. Remote}
\begin{itemize}
\item We use {\em Local memory} to indicate the memory of the same processor, and {\em remote memory} as the memory of other processors.
\item We conclude that {\em Local memory} is much much faster than {\em remote memory}.
\item This distinction applies only to distributed memory multicomputer.
\end{itemize}
\end{frame}

\begin{frame}
\frametitle{Goal}
\begin{itemize}
\item If {\em most} of the required data is nearby; then we have good performance.
\item That is, we want to ensure that most of the required data are nearby, i.e., in local memory, when we assign data to processors for computation.
\item Note that we say {\em most} because sometimes it is impossible to partition data so that all data access is local.
\end{itemize}
\end{frame}

\begin{frame}
\frametitle{Discussion}
\begin{itemize}
\item Give an example of spacial locality.
\end{itemize}
\end{frame}

\begin{frame}
\frametitle{Matrix Multiplication}
\begin{itemize}
\item We multiple matrix $A$ and $B$ and get $C$.
\item The required data of $C_{ij}$ is the $i$'th row of $A$ and $j$'th column of $B$.  
\item If we insist that the required data must be in local memory, then everything will be in one processor!
\item This is against the principle of {\em even workload distribution}.
\end{itemize}
\end{frame}

\begin{frame}
\frametitle{Proof}
\begin{itemize}
\item $C_{ij}$ has to be in the same processor as the $i$'th row of $A$ and $j$'th column of $B$.
\item $C_{kl}$ has to be in the same processor as the $k$'th row of $A$ and $l$'th column of $B$.
\item Then $C_{kj}$ has to be in the same processor as the $k$'th row of $A$ and $j$'th column of $B$.
\end{itemize}
\end{frame}

\begin{frame}
\frametitle{Proof}
\begin{itemize}
\item This implies $C_{ij}$ and $C_{kj}$ have to be in the same processor because they are in the same processor as the $j$'th column of $B$.
\item Similarly $C_{kj}$ and $C_{kl}$ have to be in the same processor because they are in the same processor as the $k$'th row of $A$.
\item We conclude that $C_{ij}$ must be in the same processor as $C_{kl}$, for any $i$, $j$, $k$, and $l$.
\item Finally, all data will be in the same processor, which is bad.
\end{itemize}
\end{frame}

\begin{frame}
\frametitle{Best Effort}
\begin{itemize}
\item If {\em most} of the required data is in local memory; then we have good performance.
\item We would like to increase the percentage of access to local memory, which is the best effort.
\item We need to carefully partition data to preserve locality.
\end{itemize}
\end{frame}

\begin{frame}
\frametitle{Communication-to-Computation Ratio}
\begin{itemize}
\item Another metric to understand the data locality is the communication-to-computation ratio.
\item The amount of computation is roughly the same throughout different data partitioning.
\item The amount of communication is proportional to remote data because local data do not incur communication.
\item If the communication-to-computation ratio is small, we have small communication overheads, which means we have good data locality.
\end{itemize}
\end{frame}

\begin{frame}
\frametitle{Discussion}
\begin{itemize}
\item Give an example of good locality and another example of bad locality for the same problem, due to different partitioning methods.
\end{itemize}
\end{frame}


\begin{frame}
\frametitle{Surface to Volume Ratio}
\begin{itemize}
\item Sometimes we use a {\em surface-to-volume} ratio to explain communication-to-computation ratio.
\item We now consider the entire data as an object, and data partitioning is a way to cut the object into pieces.
\end{itemize}
\end{frame}


\begin{frame}
\frametitle{Neighbors}
\begin{itemize}
\item In many computations, the required data are those {\em neighboring} data.
\begin{itemize}
\item In an array, the neighboring data for an array element have indices differing from the element by 1.
\item In a graph, the neighboring data are those nodes that are adjacent to the node.
\item In a graphic computation, the neighboring data for a pixel are those that are adjacent to it.
\end{itemize}
\end{itemize}
\end{frame}


\begin{frame}
\frametitle{Neighbors}
\begin{itemize}
\item In a table for dynamic programming, the value of an element is usually determined by those elements that have indices differing from the element by 1.
\item In a page ranking algorithm, the value of a node is a function of the neighboring nodes.
\item In a graphic relaxing problem, the new value of a pixel is a function of the eight neighbors.
\end{itemize}
\end{frame}


\begin{frame}
\frametitle{Discussion}
\begin{itemize}
\item Give an example of computation that uses neighbors.
\end{itemize}
\end{frame}

\begin{frame}
\frametitle{Pieces}
\begin{itemize}
\item We can use the {\em volume} of a piece to represent the number of data in a piece, which in turn represents the amount of {\em computation}.
\begin{itemize}
\item We assume that amount of workload is about the same for all data.
\end{itemize}
\item We can also use the {\em surface area} of a piece to represent the number of {\em required data} in a piece, which in turn represents the amount of {\em communication}.
\begin{itemize}
\item We assume that the required data are on the surface of the
  pieces.
\end{itemize}
\end{itemize}
\end{frame}

\begin{frame}
\frametitle{Surface-to-volume Ratio}
\begin{itemize}
\item Now we can easily relate the communication-to-computation ratio
  to the surface-to-volume ratio.
\item We want to have small communication-to-computation ratio, then
  we must partition data into pieces that have small
  surface-volume-ratio.
\begin{itemize}
\item Surface area is communication.
\item Volume is computation.
\end{itemize}
\end{itemize}
\end{frame}

\begin{frame}
\frametitle{Discussion}
\begin{itemize}
\item Give an example of surface-to-volume ratio.  If an object has a
  large surface-to-volume ratio, is it easier, or harder, to coll
  down?  How does that relate to communication costs?
\end{itemize}
\end{frame}

\begin{frame}
\frametitle{An Example}
\begin{itemize}
\item We are given a matrix of 32 by 32 by 32, and we would like to
  update each cell to be the average of its six neighbors with 8
  processors.
\item We have two choices.
\begin{itemize}
\item We cut the matrix into eight 16 by 16 by 16 cubes.
\item We cut the matrix into eight 4 by 32 by 32 slates.
\end{itemize}
\end{itemize}
\end{frame}

\begin{frame}
\frametitle{Cubes}
\begin{itemize}
\item The volume of a cube is 16 by 16 by 16 = $4k$.
\item The surface area of a cube is $6 \times 16 \times 16 = 1.5k$.
\item The surface to volume ratio is $1.5/4 = 3/8$.
\item This means for the computation on each data the processors needs
  to access remote memory $3/8$ times.
\end{itemize}
\end{frame}

\begin{frame}
\frametitle{Slates}
\begin{itemize}
\item The volume of a slate is 4 by 32 by 32 = $4k$.
\item The surface area of a slate is $2 \times 32 \times 32 + 4 \times
  32 \times 4 = 2.5k$.
\item The surface to volume ratio is $2.5/4 = 5/8$.
\item This means for the computation on each data the processors needs
  to access remote memory $5/8$ times, which is more than the $3/8$
  while cutting into cubes.
\end{itemize}
\end{frame}

\begin{frame}
\frametitle{Lessons}
\begin{itemize}
\item The surface-to-volume ratio is a reasonable estimate of the communication-to-computation ratio.
\item It is intuitive to partition the data into chunks to minimize the surface, i.e., communication.  
\item For example, if we partition the data into a checkerboard pattern, the surface-to-volume ratio will be huge, and data locality will be poor.
\end{itemize}
\end{frame}


\begin{frame}
\frametitle{Discussion}
\begin{itemize}
\item Describe the difference in sizes of similar animals that live in tropical or Arctic area.
\end{itemize}
\end{frame}

\section{Efficient Implementation}

\begin{frame}
\frametitle{Efficiency}
\begin{itemize}
\item How to synchronize processors efficiently?
\begin{itemize}
\item Global synchronization
\item Point-to-point synchronization
\end{itemize}
\item How to transfer data efficiently?
\begin{itemize}
\item Batch mode message passing
\item Overlap communication with computation
\item Explore memory hierarchy
\end{itemize}
\end{itemize}
\end{frame}

\subsection{Synchronization}

\begin{frame}
\frametitle{Global Synchronization}
\begin{itemize}
\item Reduction
\begin{itemize}
\item Every processor has a value for the solution of its sub-problem,
  and we want to compute the {\em sum} of these values.
\item Every processor has a value for the solution of its sub-problem,
  and we want to compute the {\em minimum} of these values.
\item A reduction also serves as a barrier synchronization.
\end{itemize}
\item Barrier synchronization
\begin{itemize}
\item One can think of a barrier synchronization as a special form of
  reduction in which no value is exchanged.
\end{itemize}
\end{itemize}
\end{frame}

\begin{frame}
\frametitle{Tree Optimization}
\begin{itemize}
\item We can ask a processor to coordinate the synchronization.
\begin{itemize}
\item Inherent sequential and the coordinator is the bottleneck.
\end{itemize}
\item Or we can organize the process as a tree. 
\begin{itemize}
\item We partition the processors into two subsets.
\item Two subsets recursively synchronize themselves {\em in
  parallel}.
\item Finally the two subsets synchronize with each other.
\item More details in lectures later.
\end{itemize}
\end{itemize}
\end{frame}

\begin{frame}
\frametitle{Two Party Synchronization}
\begin{itemize}
\item In a multiprocessor environment, the critical section or semaphore may not be the best synchronization solution.
\item Unlike in a uni-processor environment, the critical section or semaphore overheads are very high in a multiprocessor environment.
\item We sometimes prefer spin-locks in a multiprocessor environment, e.g., Linux kernel data structure.
\end{itemize}
\end{frame}

\subsection{Data Transfer}

\begin{frame}
\frametitle{Transfer Efficiency}
\begin{itemize}
\item In many low level parallel programming environment, (e.g. OpenCL, CUDA, or MPI) the programmers can explicit control how data is transferred among professors.
\item In these environments the programmer can apply the following techniques to improve data transfer efficiency.
\begin{itemize}
\item Batch mode message sending
\item Overlap computation with communication
\item Explore memory hierarchy 
\end{itemize}
\end{itemize}
\end{frame}

\begin{frame}
\frametitle{Batch Mode}
\begin{itemize}
\item Many message passing system is on top of network protocol like TCP/IP.
\item These protocol has a fixed start-up overhead, e.g., to establish a connection in TCP/IP.
\item If we send a large number of data through a connection, then the start-up overhead is amortized among the data begin transferred, which means we should transfer data in large quantity.
\end{itemize}
\end{frame}

\begin{frame}
\frametitle{Overlap Communication with Computation}
\begin{itemize}
\item It is beneficial to have many threads so that when a thread is waiting for data, other threads can use CPU resources for computation.
\item For example, in GPU, many running threads can hide memory latency, i.e., when a thread is waiting for memory, other threads can use ALU for computations.
\item This requires many threads and a flexible scheduler to schedule them.  
\item This relieves the burden of cache.
\item More details in later lectures.
\end{itemize}
\end{frame}

\begin{frame}
\frametitle{Explore Memory Hierarchy}
\begin{itemize}
\item In some parallel programming environments (e.g., CUDA and OpenCL), the programmer can move data with the memory hierarchy.
\item The processing units of GPU have fast and small local memory and share a slow and large global memory.
\item CUDA and OpenCL programmers must {\em explicitly} move the data between the global and local memory to achieve performance.  
\item This is a tedious and error-prone process.
\item More details on later lectures.
\end{itemize}
\end{frame}

\end{CJK}
\end{document}

