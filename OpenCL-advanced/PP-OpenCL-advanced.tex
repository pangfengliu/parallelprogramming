\documentclass{beamer}
\usetheme{Warsaw}

\usepackage{color}
\usepackage{listings}
\usepackage{url}
\usepackage{pgf}

% to specify the size of fonts in lst environment

\lstset{showstringspaces=false}
\lstset{language=C}
\lstset{basicstyle=\tt}
\lstset{keywordstyle=\color{blue}}
\lstset{emphstyle=\color{red}}
\lstset{numbers=left}
\lstset{frame=single}
\lstset{rangeprefix=\/*\ }
\lstset{rangesuffix=\ *\/}
\lstset{includerangemarker=false}

%% Program listing macro for the first page with default setting.  

\newcommand{\programlisting}[2]
{\renewcommand{\lstlistingname}{Example}
\lstinputlisting[caption={#2}]{#1.c}}

\newcommand{\programlistingoption}[3]
{\renewcommand{\lstlistingname}{Example}
\lstinputlisting[caption={(#1.c) #2},#3]{#1.c}}

\newcommand{\executionlisting}[2]
{{\renewcommand{\lstlistingname}{Execution}
\lstinputlisting[caption={}]{#1.out}}}

\newcommand{\executionlistingoption}[3]
{{\renewcommand{\lstlistingname}{Execution}
\lstinputlisting[caption={},#3]{#1.out}}}

\newcommand{\filelisting}[2]
{\renewcommand{\lstlistingname}{Example}
\lstinputlisting[caption={#2}]{#1}}

\newcommand{\prototype}[2]
{{\renewcommand{\lstlistingname}{Prototype}
\lstinputlisting[caption={#2}]{#1.h}}}

\newcommand{\pragma}[2]
{{\renewcommand{\lstlistingname}{Pragma}
\lstinputlisting[caption={#2}]{#1.h}}}


\begin{document}

\title{Advanced OpenCL Programming}

\author{Pangfeng Liu \\ National Taiwan University}

\begin{frame}
  \titlepage
\end{frame}


\section{Multiple Devices}

\begin{frame}
  \frametitle{Multiple Devices}
  \begin{itemize}
    \item We have been using one device for computation. 
    \item Now we will use multiple devices to solve a problem.
    \item We will use matrix multiplication as an example.
  \end{itemize}
\end{frame}

\begin{frame}
  \frametitle{Partition}
  \begin{itemize}
    \item Since we are very poor and cannot afford four GPUs, we will
      partition the data by rows (not by block) among devices.  
    \item We will partition matrix by rows, so that the memory
      assigned to a device is contiguous.
    \item Each kernel will run on a device, and compute part of the
      answers.
  \end{itemize}
\end{frame}

\begin{frame}
  \frametitle{Partition}
  \begin{itemize}
    \item We will partition $A$ and $C$ by rows, and do {\em not}
      partition $B$.
    \item For example, if we use two devices, then the first kernel
      will compute the top half of $C$, and another kernel will
      compute the bottom half of $C$.
      \item The reason for not partitioning $B$ is that all kernels
        need the entire $B$.
  \end{itemize}
\end{frame}

\begin{frame}
  \frametitle{Constants}
  \begin{itemize}
    \item {\tt MAXLOG} is the maximum number of bytes for storing
      compilation log.  More on this later.
    \item The number of device is {\tt DEVICENUM}.  We will use 2 in
      our humble installation.
    \item {\tt ITEMPERDEVICE} is the number of work item in {\tt
      NDRange} of a kernel.
  \end{itemize}
\end{frame}

\begin{frame}
  \frametitle{Constants}
  \programlistingfirst{matrixMul-time-copy-local-multidevice.c}{}{\footnotesize}{header}{main}
\end{frame}

\begin{frame}
  \frametitle{Discussion}
  \begin{itemize}
    \item If $N$ is 1024, and the number of device is 2, then how many
      works items are there in a kernel?
  \end{itemize}
\end{frame}

\begin{frame}
  \frametitle{Select Devices}
  \begin{itemize}
  \item We will select the first {\tt DEVICENUM} GPU device from our
    humble installation.
  \item The code is similar to those earlier version, except that
    now we need to make sure that the number of GPUs found is at
    least {\tt DEVICENUM}.
  \end{itemize}
\end{frame}

\begin{frame}
  \frametitle{Devices}
  \programlistingfirst{matrixMul-time-copy-local-multidevice.c}{}{\footnotesize}{getdevice}{getcontext}
\end{frame}

\begin{frame}
  \frametitle{Context}
  \begin{itemize}
    \item Since a context can consist of multiple devices, we need
      only one context.
      \item Here we include the first {\tt DEVICENUM} GPUs in the
        context.
  \end{itemize}
\end{frame}

\begin{frame}
  \frametitle{Context}
  \programlistingfirst{matrixMul-time-copy-local-multidevice.c}{}{\footnotesize}{getcontext}{commandqueue}
\end{frame}

\begin{frame}
  \frametitle{Multiple Queue}
  \begin{itemize}
    \item Recall that a command queue connects to a device.  
    \item Since we have multiple devices, we need multiple
      command queues.
    \item We put the command queues in the {\tt commandQueue} array.
    \item Note that we set the {\tt CL\_QUEUE\_PROFILING\_ENABLE} to
      enable profiling.
  \end{itemize}
\end{frame}

\begin{frame}
  \frametitle{Command Queue}
  \programlistingfirst{matrixMul-time-copy-local-multidevice.c}{}{\footnotesize}{commandqueue}{kernelsource}
\end{frame}

\begin{frame}
  \frametitle{One Context, Multiple Devices}
  \begin{itemize}
    \item Up to this point, we can image there is only one context,
      which has multiple devices.
    \item Within this context we will have one command queue to connect
      to each device.
    \item Later we will send commands to these devices.  The command
      sent into a command queue will run on the device this command
      queue connects to.
  \end{itemize}
\end{frame}

\begin{frame}
  \frametitle{Discussion}
  \begin{itemize}
    \item What will be sent into these command queues as commands?
  \end{itemize}
\end{frame}

\begin{frame}
  \frametitle{Program}
  \begin{itemize}
    \item The kernel will be sent into the command queue as commands.
    \item Before this can happen we need to compile the kernel.
    \item Recall that the ``kernel'' at this point is only a set
      of strings, which need to be compiled.
  \end{itemize}
\end{frame}

\begin{frame}
  \frametitle{Build Program}
  \begin{itemize}
    \item We call {\tt clBuildProgram} to build executable.
    \item Note that we pass {\em all} devices we want to use as
      parameters, so OpenCL will build executable for all our {\tt
        DEVICENUM} devices.
  \end{itemize}
\end{frame}

\begin{frame}
  \frametitle{Compilation Log}
  \begin{itemize}
    \item The kernel source is compiled when an OpenCL program runs.
    \item It is very inconvenient because when we run an OpenCL program
      the execution will not display compilation errors of the kernel.
    \item We will use {\tt clGetProgramBuildInfo} to show the error
      message if the compilation of the kernel fails.
  \end{itemize}
\end{frame}

\begin{frame}
  \prototypedetail{clGetProgramBuildInfo.h}{\scriptsize}
\end{frame}

\begin{frame}
  \frametitle{Parameters}
  \begin{description}
  \item [\tt program] The compiled program.
  \item [\tt device] The device you want to query.
  \item [\tt param\_name] The information you want to query.
    \item [\tt \tt param\_value\_size] The length of {\tt
      param\_value} in bytes.
  \item [\tt param\_value] The location for the query answer.
  \item [\tt param\_value\_size\_ret] The number of bytes returned
    from the query.
  \end{description}
\end{frame}

\begin{frame}
  \frametitle{Error}
  \begin{itemize}
    \item When there is an error in {\tt clBuildProgram}, we will call
      {\tt clGetProgramBuildInfo} to find out.
    \item Since we are not sure about which device causes the
      compilation error, we list the information from all of them.
    \item We query with {\tt CL\_PROGRAM\_BUILD\_LOG} for the
      compilation log, and place it into the {\tt log} buffer we
      prepared.
  \end{itemize}
\end{frame}


\begin{frame}
  \frametitle{Build Program}
  \programlistingfirst{matrixMul-time-copy-local-multidevice.c}{}{\footnotesize}{buildprogram}{createkernel}
\end{frame}

\begin{frame}
  \frametitle{Discussion}
  \begin{itemize}
    \item What is the type of the compilation log? 
    \item What does {\tt puts} do?
  \end{itemize}
\end{frame}

\begin{frame}
  \frametitle{Partition}
  \begin{itemize}
    \item Now we are ready to partition host buffers $A$, $B$ and $C$.
    \item We will create {\tt DEVICENUM} buffers for $A$, and each of
      which will be given to a kernel as a parameter for computations.
    \item We partition $A$ into sub-matrix of {\tt N / DEVICENUM} rows
      of $A$ each, and assign each partition to an OpenCL buffer.
  \end{itemize}
\end{frame}

\begin{frame}
  \frametitle{Buffer A}
  \begin{itemize}
    \item {\tt ITEMPERDEVICE} means the number of items per device.
    \item The index {\tt device} determines the starting position of a
      buffer, i.e., where the matrix $A$ will be partitioned.
  \end{itemize}
\end{frame}

\begin{frame}
  \frametitle{Buffer for A}
  \programlistingfirst{matrixMul-time-copy-local-multidevice.c}{}{\footnotesize}{createbuffer}{bufferB}
\end{frame}

\begin{frame}
  \frametitle{Buffer B}
  \begin{itemize}
    \item We do not need to partition $B$ because it will be used by
      all kernels.
  \end{itemize}
\end{frame}

\begin{frame}
  \frametitle{Buffer for B}
  \programlistingfirst{matrixMul-time-copy-local-multidevice.c}{}{\footnotesize}{bufferB}{bufferC}
\end{frame}

\begin{frame}
  \frametitle{Buffer C}
  \begin{itemize}
    \item Buffers for $C$ are similarly built as for $A$.
    \item Unlike $A$ and $B$ that will copy host buffers to device
      buffers, $C$ will use host memory directly.
  \end{itemize}
\end{frame}

\begin{frame}
  \frametitle{Buffer for C}
  \programlistingfirst{matrixMul-time-copy-local-multidevice.c}{}{\footnotesize}{bufferC}{NDRange}
\end{frame}

\begin{frame}
  \frametitle{Discussion}
  \begin{itemize}
    \item Make sure that you understand the offset calculation in $A$
      and $C$.
  \end{itemize}
\end{frame}

\begin{frame}
  \frametitle{\tt NDRange}
  \begin{itemize}
    \item According to our partition we declare {\tt NDRange} as a two
      dimension array with {\tt N / DEVICENUM} rows and {\tt N} columns.
    \item This is consistent with $C$, where each work item will
      compute an element of $C$.
    \item The work group is still {\tt BSIDE} by {\tt BSIDE} since we
      will use the previous local memory algorithm.
    \item We will have {\tt DEVICENUM} kernels, so we will need {\tt
      DEVICENUM} events for synchronization and profiling.
  \end{itemize}
\end{frame}

\begin{frame}
  \frametitle{NDRange}
  \programlistingfirst{matrixMul-time-copy-local-multidevice.c}{}{\footnotesize}{NDRange}{setarg}
\end{frame}

\begin{frame}
  \frametitle{Set Argument}
  \begin{itemize}
    \item We will launch {\tt DEVICENUM} kernels.  Each of them will
      compute {\tt N / DEVICENUM} rows and {\tt N} columns of $C$.
    \item We loop through all devices, and assign the argument order
      as $A$, $B$, and $C$.
    \item Note that we need to supply the correct $A$ and $C$ since
      there are {\tt DEVICENUM} of them.
  \end{itemize}
\end{frame}

\begin{frame}
  \frametitle{Set Arguments}
  \programlistingfirst{matrixMul-time-copy-local-multidevice.c}{}{\footnotesize}{setarg}{startkernel}
\end{frame}

\begin{frame}
  \frametitle{Kernel}
  \begin{itemize}
    \item Now we are ready to launch kernel.
    \item Note that we need to supply the command queue to the device
      as a parameters.
    \item The same kernel is launched again and again for {\tt
      DEVICENUM} times.  The only difference is the buffers $A$ and
      $C$ used in launching the kernel.
    \item We associate an event with each kernel launching.
  \end{itemize}
\end{frame}

\begin{frame}
  \frametitle{Start Kernel}
  \programlistingfirst{matrixMul-time-copy-local-multidevice.c}{}{\footnotesize}{startkernel}{waitforevent}
\end{frame}

\begin{frame}
  \frametitle{Wait for Completion}
  \begin{itemize}
    \item Since kernel launching is non-blocking. we need to wait for
      them to complete.
    \item We simply use {\tt clWaitForEvents}, and wait for all events
      to complete.
  \end{itemize}
\end{frame}
\begin{frame}
  \frametitle{Wait for Kernels}
  \programlistingfirst{matrixMul-time-copy-local-multidevice.c}{}{\footnotesize}{waitforevent}{getbase}
\end{frame}

\begin{frame}
  \frametitle{Discussion}
  \begin{itemize}
    \item How many events do we need to wait for the whole thing to
      complete?
  \end{itemize}
\end{frame}

\begin{frame}
  \frametitle{Kernel}
  \begin{itemize}
    \item The kernel function is very similar to the previous local
      memory version.
    \item We also use the constant {\tt DEVICENUM} to denote the
      number of devices.
  \end{itemize}
\end{frame}
  
\begin{frame}
  \frametitle{Kernel}
  \programlistingfirst{mul-local-multidevice-kernel.cl}{}{\footnotesize}{constant}{mul}
\end{frame}

\begin{frame}
  \frametitle{Kernel}
  \programlistingfirst{mul-local-multidevice-kernel.cl}{}{\footnotesize}{mul}{loop}
\end{frame}

\begin{frame}
  \frametitle{NDRange}
  \begin{itemize}
    \item We declare $A$ and $C$ as {\tt N / DEVICENUM} by {\tt N}
      arrays, because the domain is partitioned among devices.
    \item The rest of the code is {\em not} changed at all!
  \end{itemize}
\end{frame}

\begin{frame}
  \frametitle{Reason}
  \begin{itemize}
  \item The reason for this ``no need to change'' is that a kernel
    function is acting from a local view, i.e., it is a work item
    within a work group, so all the operations are still the same.
  \item The kernel still thinks from the point of view of blocks, and
    the related operations are done in local indices.
  \item We only need the global index while accessing $A$, $B$, and
    $C$.
  \item Since the kernel is given the corresponding parts of $A$ and
    $B$, so the operation is correct.
  \end{itemize}
\end{frame}

\begin{frame}
  \frametitle{Discussion}
  \begin{itemize}
    \item Convince yourself that the code is correct.
  \end{itemize}
\end{frame}

\begin{frame}
  \frametitle{Timing}
  \begin{itemize}
    \item We would like to know the detailed timing of the kernels.
    \item In particular we would like to prove the two kernel run {\em in
      parallel}.
    \item To do so we need the absolute time of the events from all devices.
  \end{itemize}
\end{frame}

\begin{frame}
  \frametitle{Base}
  \begin{itemize}
    \item The time returned by the event is difficult to interpret.
    \item We would like to establish a relative time for inspection.
      \item We choose the time the first kernel entering the queue as
        the base time, and report the relative time to it for ease
        understanding.
  \end{itemize}
\end{frame}

\begin{frame}
  \frametitle{Get Time}
  \programlistingfirst{matrixMul-time-copy-local-multidevice.c}{}{\footnotesize}{getbase}{gettime}
\end{frame}

\begin{frame}
  \frametitle{Get Time}
  \begin{itemize}
    \item We loop through all devices and get timing information from
      all of them.
    \item This part is the same as before.
  \end{itemize}
\end{frame}

\begin{frame}
  \frametitle{Get Time}
  \programlistingfirst{matrixMul-time-copy-local-multidevice.c}{}{\footnotesize}{gettime}{getrest}
\end{frame}

\begin{frame}
  \frametitle{Get Time}
  \programlistingfirst{matrixMul-time-copy-local-multidevice.c}{}{\footnotesize}{getrest}{printtime}
\end{frame}

\begin{frame}
  \frametitle{Relative Time}
  \begin{itemize}
    \item In addition to queuing time, submission time, and
      execution, we also report the relative time when the four
      events happened.
    \item The relative time is in nanosecond.
  \end{itemize}
\end{frame}

\begin{frame}
  \frametitle{Print Time}
  \programlistingfirst{matrixMul-time-copy-local-multidevice.c}{}{\footnotesize}{printtime}{checkandfree}
\end{frame}

\begin{frame}
  \frametitle{Demonstration}
  \begin{itemize}
    \item Run the {\tt matrixMul-time-copy-local-multidevice-cl} program.
  \end{itemize}
\end{frame}

\begin{frame}
  \frametitle{Discussion}
  \begin{itemize}
    \item Does the kernels run in parallel? Observe the results and
      give your answer.
  \end{itemize}
\end{frame}

\section{Dependency}

\begin{frame}
  \frametitle{Dependency}
  \begin{itemize}
    \item We may launch multiple kernels into multiple devices.
    \item Kernels may have dependency, and we need to ensure dependency
    among kernels.
  \end{itemize}
\end{frame}

\begin{frame}
  \frametitle{Example}
  \begin{itemize}
  \item In the following example we compute $G = (A + B) + (D + E)$,
    where all vector has length $N$.
  \item We compute $C = A + B$, then $F = D + E$, then we compute $G =
    C + F$.  The last computation must wait for the first two to
    complete.
    \item We will use one device for {\em each} of the three additions.
  \end{itemize}
\end{frame}

\begin{frame}
  \frametitle{Constants}
  \begin{itemize}
    \item We first declare the constants and variables.
  \end{itemize}
\end{frame}

\begin{frame}
  \programlistingfirst{vectorAdd-dependency.c}{}{\small}{header}{main}
\end{frame}

\begin{frame}
  \frametitle{Initialization}
  \begin{itemize}
    \item We initialize the $A$, $B$, $D$ and $E$ vectors.
  \end{itemize}
\end{frame}


\begin{frame}
  \programlistingfirst{vectorAdd-dependency.c}{}{\footnotesize}{vectors}{createbuffer1}
\end{frame}

\begin{frame}
  \frametitle{Buffers}
  \begin{itemize}
    \item We then create buffers for kernel parameters.
    \item $A$ and $B$ are read only, and need to be copied into device.
    \item $C$ is both read and write enable.
  \end{itemize}
\end{frame}

\begin{frame}
  \programlistingfirst{vectorAdd-dependency.c}{}{\scriptsize}{createbuffer1}{createbuffer2}
\end{frame}

\begin{frame}
  \frametitle{Buffers}
  \begin{itemize}
    \item Similarly we create buffers for $D$, $E$ and $F$.
  \end{itemize}
\end{frame}

\begin{frame}
  \programlistingfirst{vectorAdd-dependency.c}{}{\scriptsize}{createbuffer2}{createbuffer3}
\end{frame}

\begin{frame}
  \frametitle{Buffers}
  \begin{itemize}
    \item We create buffers for $G$.
    \item This buffer is write only and use the host memory directly,
      so the host can print it directly.
  \end{itemize}
\end{frame}

\begin{frame}
  \programlistingfirst{vectorAdd-dependency.c}{}{\scriptsize}{createbuffer3}{shape}
\end{frame}

\begin{frame}
  \frametitle{Discussion}
  \begin{itemize}
    \item What are the characteristics for read only, read and write,
      and write only buffers respectively?
  \end{itemize}
\end{frame}

\begin{frame}
  \frametitle{Buffers}
  \begin{itemize}
    \item Both {\tt NDRange} and the work group are one dimensional.
    \item The size of {\tt NDRange} is $N$.
    \item The size of a work group is 256.
  \end{itemize}
\end{frame}

\begin{frame}
  \programlistingfirst{vectorAdd-dependency.c}{}{\scriptsize}{shape}{ABC}
\end{frame}

\begin{frame}
  \frametitle{$C = A + B$}
  \begin{itemize}
    \item Now we launch the first kernel for $C = A + B$.
    \item We have already built three command queues for three devices.  
    \item We use {\tt commandQueue[0]} for the first addition.
    \item We also declared three events to denote the completion of
      the three kernels that we will launch.
  \end{itemize}
\end{frame}

\begin{frame}
  \programlistingfirst{vectorAdd-dependency.c}{}{\scriptsize}{ABC}{DEF}
\end{frame}

\begin{frame}
  \frametitle{$F = D + E$}
  \begin{itemize}
    \item Similarly  we launch the second kernel for $F = D + E$.
      \item Note that we used {\tt commandQueue[1]} and {\tt
        events[1]}.
  \end{itemize}
\end{frame}

\begin{frame}
  \programlistingfirst{vectorAdd-dependency.c}{}{\scriptsize}{DEF}{CFG}
\end{frame}

\begin{frame}
  \frametitle{$G = C + F$}
  \begin{itemize}
    \item Finally  we launch the third kernel for $G = C + F$.
    \item Note that we used {\tt commandQueue[2]} and {\tt
      events[2]}.
  \end{itemize}
\end{frame}

\begin{frame}
  \programlistingfirst{vectorAdd-dependency.c}{}{\scriptsize}{CFG}{waitforevent}
\end{frame}

\begin{frame}
  \frametitle{Wait for Events}
  \begin{itemize}
    \item We wait for all three kernel to finish.
  \end{itemize}
\end{frame}

\begin{frame}
  \programlistingfirst{vectorAdd-dependency.c}{}{\scriptsize}{waitforevent}{getbase}
\end{frame}

\begin{frame}
  \frametitle{Wait for Events}
  \begin{itemize}
  \item The rest of the code just prints the timing information, checks
    for correctness, and releases resources.
  \item Please refer to previous discussion.
  \end{itemize}
\end{frame}

\begin{frame}
  \frametitle{Demonstration}
  \begin{itemize}
    \item Run the {\tt vectorAdd-dependency-cl}.
  \end{itemize}
\end{frame}

\begin{frame}
  \frametitle{Discussion}
  \begin{itemize}
    \item Does the program produce the correct answer?  
    \item If not, what is the possible reason?
  \end{itemize}
\end{frame}

\subsection{Wait for Events}

\begin{frame}
  \frametitle{Problem}
  \begin{itemize}
    \item The previous program waits for all three kernels to
      complete, but the third kernel did {\em not} wait for the first
      two.
    \item We will use the {\em wait for events} mechanism to launch
      the third kernel.
  \end{itemize}
\end{frame}

\begin{frame}
  \prototypedetail{clEnqueueNDRangeKernel.h}{\scriptsize}
\end{frame}

\begin{frame}
  \frametitle{$G = C + F$}
  \begin{itemize}
    \item We launch the third kernel and wait for the first two events
      in the {\tt events} array.
    \item Set {\tt num\_events\_in\_wait\_list} to 2 and {\tt
      event\_wait\_list} to {\tt events}.
  \end{itemize}
\end{frame}

\begin{frame}
  \programlistingfirst{vectorAdd-dependency-correct.c}{}{\scriptsize}{CFG}{waitforevent}
\end{frame}

\begin{frame}
  \frametitle{Demonstration}
  \begin{itemize}
    \item Run the {\tt vectorAdd-dependency-correct-cl}.
  \end{itemize}
\end{frame}

\begin{frame}
  \frametitle{Discussion}
  \begin{itemize}
    \item Does the program produce the correct answer?  
    \item Observe the timing and make sure the first two kernels run
      in parallel, and the third kernel will wait for the first two.
  \end{itemize}
\end{frame}

\subsection{\tt clFinish}

\begin{frame}
  \frametitle{\tt clFinish}
  \begin{itemize}
    \item The host can also explicitly wait for the first two kernels
      to finish before launching the third kernel by {\tt clFinish}.
    \item We just need to wait for the command queues to the first two
      devices to become empty.
  \end{itemize}
\end{frame}

\begin{frame}
  \programlistingfirst{vectorAdd-dependency-clfinish.c}{}{\scriptsize}{CFG}{waitforevent}
\end{frame}

\begin{frame}
  \frametitle{Demonstration}
  \begin{itemize}
    \item Run the {\tt vectorAdd-dependency-clfinish-cl} program.
  \end{itemize}
\end{frame}

\begin{frame}
  \frametitle{Discussion}
  \begin{itemize}
    \item Does the program produce the correct answer?  
    \item Observe the timing and make sure the first two kernels run
      in parallel, and the third kernel will wait for the first two.
    \item Compare the timing results between {\tt
      vectorAdd-dependency-correct-cl} and {\tt
      vectorAdd-dependency-clfinish-cl}, especially in when the last
      kernel joined the command queue.
  \end{itemize}
\end{frame}

\section{Group Size}

\begin{frame}
  \frametitle{Group Size}
  \begin{itemize}
  \item The previous matrix multiplication program has extremely good
    performance due to two reasons.
    \begin{itemize}
      \item It uses local memory to speed up data access.
      \item It has a larger group size (256).
    \end{itemize}
    \item Now we would like to study the effects of group size on
      performance.
  \end{itemize}
\end{frame}

\begin{frame}
  \frametitle{Compute Units}
  \begin{itemize}
    \item A device has only a limited number of compute units.
    \item The work items of a work group will occupy a compute unit.
    \item For a given number of work items, if the work group size is
      small, then the number of work groups becomes large, and some
      work groups will wait for compute unit.
    \item In other words, a small work group size limits the the
      number of work items that we can process in parallel.
  \end{itemize}
\end{frame}

\begin{frame}
  \frametitle{Classroom}
  \begin{itemize}
    \item A school has only a limited number of classrooms.
    \item The students are divided into group, and one group will
      occupy a classroom.
    \item For a given number of students, if the group size is small,
      then the number of groups becomes large, and some groups will
      wait for classroom.
    \item In other words, a small group size limits the the number of
      students that can study in parallel.
  \end{itemize}
\end{frame}

\begin{frame}
  \frametitle{Discussion}
  \begin{itemize}
    \item Give your own example of this phenomenon.
  \end{itemize}
\end{frame}

\begin{frame}
  \frametitle{Compute Unit Number}
  \begin{itemize}
    \item We need only one GPU device.
    \item We call {\tt clGetDeviceInfo} with {\tt
      CL\_DEVICE\_MAX\_COMPUTE\_UNITS} to get the number of compute
      units on GPU.
    \item We need the number of compute units to determine the work
      group size.
  \end{itemize}
\end{frame}

\begin{frame}
  \programlistingfirst{vectorAdd-groupsize.c}{}{\footnotesize}{getdevice}{getcontext}
\end{frame}


\begin{frame}
  \frametitle{Group Size}
  \begin{itemize}
    \item We fix the number of work items to $N$, and vary the size of
      the work group from 1 to 256.
    \item We wait for the event before launching the next kernel.
    \item We will record the timing information into a file {\tt
      vectorAdd-grouopsize.dat}.
  \end{itemize}
\end{frame}

\begin{frame}
  \programlistingfirst{vectorAdd-groupsize.c}{}{\scriptsize}{setNDRange}{getbase}
\end{frame}

\begin{frame}
  \frametitle{Get Timing Information}
  \begin{itemize}
    \item The rest of the code will retrieve timing information from
      the event.
    \item We output the group size and the execution time into a file
      {\tt vectorAdd-grouopsize.dat}.
  \end{itemize}
\end{frame}

\begin{frame}
  \programlistingfirst{vectorAdd-groupsize.c}{}{\scriptsize}{printtime}{checkandfree}
\end{frame}

\begin{frame}
  \frametitle{Demonstration}
  \begin{itemize}
    \item We plot the kernel execution time for different group sizes.
    \item The x-xis is in log scale.
  \end{itemize}
\end{frame}

\begin{frame}
  \frametitle{Execution Time}
  \centerline{\pgfimage[width=0.9\textwidth]{vectorAdd-groupsize.png}}
\end{frame}

\begin{frame}
  \frametitle{Discussion}
  \begin{itemize}
    \item Observe the execution time and draw you conclusion about the
      effects of group size on performance.
  \end{itemize}
\end{frame}

\end{document}
