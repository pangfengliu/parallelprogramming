\documentclass{beamer}
\usetheme{Warsaw}

\usepackage{color}
\usepackage{listings}
\usepackage{url}
\usepackage{booktabs}
\usepackage{pgf}
\usepackage{multicol}
\usepackage{algorithm}
\usepackage{algorithmic}

% to specify the size of fonts in lst environment

\lstset{showstringspaces=false}
\lstset{language=C}
\lstset{basicstyle=\tt}
\lstset{keywordstyle=\color{blue}}
\lstset{emphstyle=\color{red}}
\lstset{numbers=left}
\lstset{frame=single}
\lstset{rangeprefix=\/*\ }
\lstset{rangesuffix=\ *\/}
\lstset{includerangemarker=false}

%% Program listing macro for the first page with default setting.  

\newcommand{\programlisting}[2]
{\renewcommand{\lstlistingname}{Example}
\lstinputlisting[caption={#2}]{#1.c}}

\newcommand{\programlistingoption}[3]
{\renewcommand{\lstlistingname}{Example}
\lstinputlisting[caption={(#1.c) #2},#3]{#1.c}}

\newcommand{\executionlisting}[2]
{{\renewcommand{\lstlistingname}{Execution}
\lstinputlisting[caption={}]{#1.out}}}

\newcommand{\executionlistingoption}[3]
{{\renewcommand{\lstlistingname}{Execution}
\lstinputlisting[caption={},#3]{#1.out}}}

\newcommand{\filelisting}[2]
{\renewcommand{\lstlistingname}{Example}
\lstinputlisting[caption={#2}]{#1}}

\newcommand{\prototype}[2]
{{\renewcommand{\lstlistingname}{Prototype}
\lstinputlisting[caption={#2}]{#1.h}}}

\newcommand{\pragma}[2]
{{\renewcommand{\lstlistingname}{Pragma}
\lstinputlisting[caption={#2}]{#1.h}}}


\begin{document}

\title{Thread}

\author{Pangfeng Liu \\ National Taiwan University}

\begin{frame}
\titlepage
\end{frame}

\section{Process and Thread} 

\begin{frame}
\frametitle{Process and Threads}
\begin{itemize}
\item A process is a program in execution.
\item A computer may have multiple processes running simultaneously.
\item A thread is an execution instance of a process.
\item A process may have multiple threads running simultaneously.
\end{itemize}
\end{frame}

\begin{frame}
\frametitle{UNIX Process}
A UNIX process has the following
information~\footnote{\url{https://computing.llnl.gov/tutorials/pthreads}}.
\begin{itemize}
\item Process ID, process group ID, user ID, and group ID
\item Environment variables
\item Working directory
\item Program instructions
\item Registers
\item File descriptors
\item Signal actions
\end{itemize}
\end{frame}

\begin{frame}
\frametitle{UNIX Process}
A UNIX process may place the data in the following sections.
\begin{itemize}
\item Stack
\begin{itemize}
\item Variables declared within functions.
\end{itemize}
\item Heap
\begin{itemize}
\item Storage allocated dynamically (e.g., malloc).
\end{itemize}
\item Data 
\begin{itemize}
\item Variables declared globally.
\end{itemize}
\end{itemize}
\end{frame}

\begin{frame}
\frametitle{UNIX Process}
\centerline{\pgfimage[height=0.7\textheight]{process}}
\footnote{\url{https://computing.llnl.gov/tutorials/pthreads/images/process.gif}}
\end{frame}

\begin{frame}
\frametitle{Discussion}
\begin{itemize}
\item How to create a process in a UNIX environment?
\item How to kill a process in a UNIX environment?
\end{itemize}
\end{frame}

\begin{frame}
\frametitle{Thread}
\begin{itemize}
\item A thread executes within a process, and has its own resources --
  program text, program counter, stack, private variables, etc.
\item A thread runs on a core, and multiple threads can run on
  multiple cores simultaneously.
\item We use threads to explore parallelism.
\end{itemize}
\end{frame}

\begin{frame}
\frametitle{Parallelism}
\begin{itemize}
\item We want to express parallelism with multi-threading on a
  multiprocessor because multiple threads can run on multiple cores
  simultaneously.
\item We do {\em not} use processes since the communication among them
  require expensive inter-process-communication (IPC).
\begin{itemize}
\item Threads within a process can easily share variables in a shared
  memory multiprocessor.
\end{itemize}
\item We do {\em not} use processes since the process creation is expensive.
\begin{itemize}
\item Thread creation is much cheaper.
\end{itemize}
\end{itemize}
\end{frame}

\begin{frame}
\frametitle{Thread}
\begin{itemize}
\item Threads exist within a process and share some of the process
  resources.
\item Threads have their own independent flows of control.
\item Threads die if the process dies.
\item Threads are "lightweight" because most of the overhead has
  already been accomplished through the creation of its process.
\end{itemize}\footnote{\url{https://computing.llnl.gov/tutorials/pthreads}}
\end{frame}


\begin{frame}
\frametitle{Threads within a Process}
\centerline{\pgfimage[height=0.7\textheight]{thread.png}}
\footnote{\url{https://computing.llnl.gov/tutorials/pthreads/images/thread.gif}}
\end{frame}

\begin{frame}
\frametitle{Memory Model} 
\begin{itemize}
\item All threads have access to the same global shared memory and
  programmers are responsible for synchronizing access (protecting)
  them.
\item Threads also have their own private data for their own private
  usage.
\end{itemize}
\end{frame}

\begin{frame}
\frametitle{Shared Memory}
\centerline{\pgfimage[height=0.7\textheight]{sharedMemoryModel.png}}
\footnote{\url{https://computing.llnl.gov/tutorials/pthreads/images/sharedmemoryModel.gif}}
\end{frame}


\section{Pthread}

\begin{frame}
\frametitle{Pthread}
\begin{itemize}
\item Historically, hardware vendors have implemented their own
  proprietary versions of threads. 
\item These implementations differed substantially from each other
  making it difficult for programmers to develop portable threaded
  applications.
\end{itemize}
\end{frame}

\begin{frame}
\frametitle{Pthread}
\begin{itemize}
\item In order to take full advantage of the capabilities provided by
  threads, a standardized programming interface was
  required\footnote{\url{https://computing.llnl.gov/tutorials/pthreads}}.
\item For UNIX systems, this interface has been specified by the IEEE
  POSIX 1003.1c standard (1995).
\item Implementations adhering to this standard are referred to as
  POSIX threads, or Pthreads.
\item Most hardware vendors now offer Pthreads in addition to their
  proprietary API's.
\end{itemize}
\end{frame}

\begin{frame}
\frametitle{Discussion}
\begin{itemize}
\item Describe a ``standard'' in computer science, and why is it
  important.
\end{itemize}
\end{frame}

\begin{frame}
\frametitle{Hello}
\programlistingoption{hello}{Main program}{linerange=main-end}
\end{frame}

\begin{frame}
\frametitle{Hello}
\programlistingoption{hello}{Print}{linerange=print-main}
\end{frame}

\begin{frame}
\frametitle{Demonstration}
\begin{itemize}
\item Run the hello.c program.
\end{itemize}
\end{frame}

\begin{frame}
\frametitle{Implementation}
\begin{itemize}
\item Implement a program that prints 10 lines of ``hello''.
\end{itemize}
\end{frame}

\begin{frame}
\frametitle{Discussion}
\begin{itemize}
\item Describe the compiling environment you used.
\end{itemize}
\end{frame}


\begin{frame}
\frametitle{Pthread Concepts}
\begin{itemize}
\item Initially there is only one {\em main thread}.
\item The main thread can spawn other threads.  For ease of
  explanation we will refer to them as the {\em spawned} threads of
  the main thread.
\item For ease of explanation we will also use {\em child} threads to
  refer to those threads spawned by the main thread.
\item You have been programming the main thread {\em only}.
\end{itemize}
\end{frame}


\begin{frame}
\frametitle{Pthread Steps}
\begin{itemize}
\item Include the necessary header file {\tt pthread.h}.
\item Declare a variable of type {\tt pthread\_t} to store the
  identifier for the thread you will create.
\item Call {\tt pthread\_create} creates a thread with the type
  {\tt pthread\_t} variable declared previously.
\item Call {\tt pthread\_exit} when the main thread or the spawn
  thread wants to exit.
\end{itemize}
\end{frame}


\begin{frame}
\frametitle{Hello}
\programlistingoption{hello-pthread}{Declaration}{linerange=declaration-print,emph={pthread}}
\begin{itemize}
\item Must include {\tt pthread.h}.
\end{itemize}
\end{frame}

\subsection{\tt pthread\_create}

\begin{frame}
\frametitle{\tt pthread\_create}
\programlisting{pthread_create}{\tt pthread\_create}
\end{frame}

\begin{frame}
\frametitle{\tt pthread\_create Parameters}
\begin{description}[l]
\item[\tt thread] A {\tt pthread\_t} pointer to the created thread.
\item[\tt attr] A {\tt pthread\_type\_t} pointer to the attribute of the
  thread.  We can set it to NULL for now.
\end{description}
\end{frame}

\begin{frame}
\frametitle{\tt pthread\_create Parameters}
\begin{description}[l]
\item[\tt start\_routinue] Thread starting routine.  The routine must
  have {\tt void *routine(void *)} prototype. i.e., it expects a
  pointer to {\tt void} as the only parameter, and returns a pointer
  to {\tt void}.
\item[\tt arg] A pointer to an optional parameter to the spawned
  thread.  We set it to the thread index {\tt t} in our example.
\item[\tt return value] Return 0 if success, and a non-zero error code
  on error.
\end{description}
\end{frame}

\subsection{\tt pthread\_exit}

\begin{frame}
\frametitle{\tt pthread\_exit}
\programlisting{pthread_exit}{\tt pthread\_exit}
\begin{description}[l]
\item[\tt value] A thread can use this pointer to sent information
  back to its parent thread.  We can set it to {\tt NULL} for now
  because the threads will not send anything back to the main thread.
\end{description}
\end{frame}

\begin{frame}
\frametitle{Discussion}
\begin{itemize}
\item Name the functions that create and destroy pthreads respectively.
\end{itemize}
\end{frame}

\begin{frame}
\frametitle{Hello with Pthread}
\programlistingoption{hello-pthread}{Main program}{linerange=main-end,basicstyle={\scriptsize \tt},emph={threads,pthread_t,pthread_create,pthread_exit}}
\end{frame}

\begin{frame}
\frametitle{Main Program}
\begin{itemize}
\item We can use the argument {\tt arg} to pass information to the
  spawned threads.
\item In our example we pass the index {\tt t} to the spawned thread
  as thread index.
\item Note that the type of {\tt arg} is {\tt void *} since we do not
  know what kind of parameter will be passed into a thread.
\end{itemize}
\end{frame}

\begin{frame}
\frametitle{Hello with Pthread}
\programlistingoption{hello-pthread}{Print}{linerange=print-main,emph={pthread_exit,threadid}}
\end{frame}

\begin{frame}
\frametitle{\tt printHello}
\begin{itemize}
\item This is the routine each spawned thread will run.
\item Each thread receives a pointer to the thread index ({\tt
  thread\_id}) from the main thread, and prints it.
\item We need to cast the type of the parameter back to {\tt (int *)},
  very much like the case of qsort.
\item We do not have anything to send back to the main thread, so we
  use {\tt NULL} in {\tt pthread\_exit(NULL)}.
\end{itemize}
\end{frame}

\begin{frame}
\begin{itemize}
\frametitle{Compile and Link}
\item You need to link your program with the pthread library.
\item You can use gcc to compile and link you program with pthread
  library.  Note that the library must be at the end of the gcc
  command.
\begin{itemize}
\item {\tt gcc program.c -o program -lpthread}
\end{itemize}
\end{itemize}
\end{frame}

\begin{frame}
\frametitle{Demonstration}
\begin{itemize}
\item Run the hello-pthread.c program.
\end{itemize}
\end{frame}

\begin{frame}
\frametitle{Discussion}
\begin{itemize}
\item Is the answer correct? If not what is the reason?
\end{itemize}
\end{frame}

\begin{frame}
\frametitle{Hello with Pthread}
\programlistingoption{hello-pthread-correct}{main}{linerange=main-end,emph={pthread_exit,threadid,threadIndex},basicstyle={\tt \scriptsize}}
\end{frame}

\begin{frame}
\begin{itemize}
\frametitle{Main Thread Variables}
\item We use an array {\tt threadIndex} to keep the thread index for
  all threads.
\item We then pass the address of the index to the spawned threads.
  This prevents the threads from accessing the same memory address in
  the previous example.
\item From this example we know that the spawned threads can access
  the memory of the main thread.
\end{itemize}
\end{frame}

\begin{frame}
\frametitle{Demonstration}
\begin{itemize}
\item Run the hello-pthread-correct.c program.
\item Is the answer correct?
\item Is there timing constraints between these threads?
\end{itemize}
\end{frame}

\begin{frame}
\frametitle{Implementation}
\begin{itemize}
\item Implement a pthread program that prints 10 lines of ``hello''.
  You should use {\tt pthread\_create} to create 10 threads to do
  this, where each thread prints an index from 0 to 9.
\end{itemize}
\end{frame}

\begin{frame}
\frametitle{Discussion}
\begin{itemize}
\item Does your program function properly if you remove the {\tt
  pthread\_exit} from the main program?  Explain the reason if it does
  not.
\end{itemize}
\end{frame}

\begin{frame}
\frametitle{Eight Queen} 
\begin{itemize}
\item Place eight queen on a chessboard so that no queen can attack
  any other
  queen.\footnote{\url{http://support.sas.com/documentation/cdl/en/orcpug/59630/HTML/default/images/queens.png}}
\item We would like to know the number of such solutions.
\end{itemize}
\centerline{\pgfimage[height=0.55\textheight]{queens}}
\end{frame}

\begin{frame}
\frametitle{Solution}
\begin{itemize}
\item Use an array {\tt position} to store the positions of $n$
  queens.
\item The $i$-th element of the array stores the row index of the
  queen at the $i$-column.
\item Receive the size of the board $n$ as the second command line
  argument.
\item Use a recursive function {\tt queen} to compute the number of
  solutions.
\end{itemize}
\end{frame}

\begin{frame}
\frametitle{Headers}
\programlistingoption{queen}{Headers}{linerange=begin-ok}
\end{frame}

\begin{frame}
\frametitle{Headers} 
\begin{itemize}
\item {\tt stdlib.h} for {\tt atoi}.
\item {\tt MAXN} is the maximum size of the board.
\end{itemize}
\end{frame}

\begin{frame}
\frametitle{Eight Queen}
\programlistingoption{queen}{Main program}{linerange=main-end}
\end{frame}

\begin{frame}
\frametitle{\tt main} 
\begin{itemize}
\item {\tt position} for queen positions.
\item {\tt n} is the size of the board.
\item {\tt queen} will return the number of solutions.
\end{itemize}
\end{frame}

\begin{frame}
\frametitle{Eight Queen}
\programlistingoption{queen}{queen}{linerange=queen-main}
\end{frame}

\begin{frame}
\frametitle{\tt queen}
\begin{itemize}
\item The {\tt queen} function computes the number of solutions.
\item The {\tt queen} uses a parameter {\tt next} to keep track of the
  column number it wishes to place a queen next.
\item If {\tt next} is already $n$ then we have placed all $n$ queens.
\item Otherwise we try all rows and sum up the number of solutions
  from all possible row placements.
\end{itemize}
\end{frame}

\begin{frame}
\frametitle{Eight Queen}
\programlistingoption{queen}{ok}{linerange=ok-queen}
\end{frame}

\begin{frame}
\frametitle{\tt ok}
\begin{itemize}
\item The {\tt queen} function determines if we could place a queen
 at the end of {\tt position}.
\item The newly added queen is at row {\tt position[next]}, column
  {\tt next}.
\item if any previously placed queen is at the same row or at the
  diagonal of the newly placed queen, the function returns 0.
  Otherwise it return 1.
\end{itemize}
\end{frame}

\begin{frame}
\frametitle{Demonstration}
\begin{itemize}
\item Run and time the queen.c program.
\end{itemize}
\end{frame}

\begin{frame}
\frametitle{Implementation}
\begin{itemize}
\item Implement a sequential program that solves the eight queen
  problem.
\end{itemize}
\end{frame}

\begin{frame}
\frametitle{Discussion}
\begin{itemize}
\item Can you think of any simple optimization that will speed up your
  program?
\end{itemize}
\end{frame}

\begin{frame}
\frametitle{$n$ Queen with pthread} Now we want to parallelize the
previous program with pthread.
\begin{itemize}
\item We create $n$ threads, and use one thread to search each of the
  $n$ subtrees when we place the first queen at all $n$ possible cells
  in the first column.
\item All threads can run in parallel.  
\item Right now we just let each thread report the number of answers
  it finds, and do not sum them.
\end{itemize}
\end{frame}

\begin{frame}
\frametitle{Issues}
\begin{itemize}
\item How do all thread know the size of the board?
  \begin{itemize}
  \item We declare a global {\tt n}.
  \end{itemize}
\item How does a thread know which subtree it should solve?
  \begin{itemize}
  \item We pass this information to the spawned threads by the {\tt arg}
    parameter in {\tt pthread\_create}.
  \end{itemize}
\item Can these threads share a board?
  \begin{itemize}
  \item No, they need their own {\tt position}.
  \end{itemize}
\item Do these threads need to communicate with each other?
  \begin{itemize}
  \item No, because right now we do not sum the numbers of solutions.
  \end{itemize}
\end{itemize}
\end{frame}


\begin{frame}
\frametitle{Headers}
\programlistingoption{queen-pthread}{Headers}{linerange=begin-ok}
\end{frame}

\begin{frame}
\frametitle{Headers} 
\begin{itemize}
\item {\tt pthread.h} for pthread functions.
\item {\tt stdlib.h} for {\tt atoi}.
\item {\tt MAXN} is the maximum size of the board.
\item A global {\tt n} for all threads.
\end{itemize}
\end{frame}

\begin{frame}
\frametitle{Eight Queen with Pthread}
\programlistingoption{queen-pthread}{Main program}{linerange=main-end,basicstyle={\scriptsize \tt},emph={pthread_create,position,calloc}}
\end{frame}

\begin{frame}
\frametitle{\tt main}
\begin{itemize}
\item We call {\tt calloc} to allocate an array and keep the starting
  address in {\tt position}.
\item We set the first element of {\tt position} according to the
  thread index, so that each thread searches its own subtree.
\item We pass the {\tt position} to {\tt pthread\_create} so that the
  threads can access their own arrays.
\item From this example we know that all threads share the global
  variables and heap space.
\end{itemize}
\end{frame}

\begin{frame}
\frametitle{\tt goQueen}
\programlistingoption{queen-pthread}{\tt goqueen}{linerange=go-main,basicstyle={\footnotesize \tt},emph={position}}
\end{frame}

\begin{frame}
\frametitle{\tt goQueen}
\begin{itemize}
\item {\tt goQueen} is the starting function of spawned threads.
\item {\tt goQueen} uses its own pointer {\tt petition} to point to
  the array in the heap.  Some casting is required.
\item {\tt goQueen} calls {\tt queen} with {\tt next} set to 1.
\item All threads can get the board size from the global variable {\tt
  n}.
\end{itemize}
\end{frame}

\begin{frame}
\frametitle{Demonstration}
\begin{itemize}
\item Run and time the queen-pthread.c program.
\end{itemize}
\end{frame}

\begin{frame}
\frametitle{Implementation}
\begin{itemize}
\item Implement a parallel program that solves the $n$ queen problem
  with pthread.  Each thread only reports the number of solutions it
  finds and no summation is required.
\item Measure the speedup and efficiency.
\end{itemize}
\end{frame}

\begin{frame}
\frametitle{Discussion}
\begin{itemize}
\item Why do we call queen with 1?
\item What are the speedup and efficiency you are getting?
\item Is the workload evenly distributed?
\end{itemize}
\end{frame}

\begin{frame}
\frametitle{Total Number}
\begin{itemize}
\item Now we want to compute the total number of solutions, instead of
  individual number of solution from each thread.
\item We use a global variable {\tt numSolution} to store the sum of
  the numbers of solutions from all threads.  This variable is
  initialized to 0 automatically.
\end{itemize}
\end{frame}

\begin{frame}
\frametitle{Headers}
\programlistingoption{queen-pthread-sum}{Headers}{linerange=begin-ok,emph={numSolution}}
\end{frame}

\begin{frame}
\frametitle{Headers} 
\begin{itemize}
\item {\tt pthread.h} for pthread functions.
\item {\tt stdlib.h} for {\tt atoi}.
\item {\tt MAXN} is the maximum size of the board.
\item A global {\tt n} for all threads.
\item A global {\tt numSolution} for the total number of solutions.
\end{itemize}
\end{frame}

\begin{frame}
\frametitle{Eight Queen with Pthread}
\programlistingoption{queen-pthread-sum}{Main program}{linerange=main-end,basicstyle={\scriptsize \tt},emph={pthread_create,numSolution}}
\end{frame}

\begin{frame}
\frametitle{\tt main}
\begin{itemize}
\item We print the number of solutions {\tt numSolution} before we
  call {\tt pthread\_exit}.
\item If we place the {\tt printf} after {\tt pthread\_exit}, it will
  not be executed.
\end{itemize}
\end{frame}

\begin{frame}
\frametitle{\tt goQueen}
\programlistingoption{queen-pthread-sum}{\tt goqueen}{linerange=go-main,basicstyle={\footnotesize \tt},emph={numSolution}}
\end{frame}

\begin{frame}
\frametitle{\tt goQueen}
\begin{itemize}
\item {\tt goQueen} reports the number of solution {\tt queen} finds.
\item {\tt goQueen} also adds the number of solutions computed by {\tt
  queen} to the global variable {\tt numSolution}.
\end{itemize}
\end{frame}

\begin{frame}
\frametitle{Demonstration}
\begin{itemize}
\item Run and time the queen-pthread-sum.c program.
\end{itemize}
\end{frame}

\begin{frame}
\frametitle{Discussion}
\begin{itemize}
\item Does the program produce correct output?
\end{itemize}
\end{frame}

\subsection{\tt pthread\_join}

\begin{frame}
\frametitle{Synchronization}
\begin{itemize}
\item The problem of the previous program is that the main thread did
  {\em not} wait for all spawned threads to complete.
\item We need a {\em barrier synchronization}, and after that the main
  thread can get the correct total number of solutions in {\tt
    numSolution}.
\item The main thread can wait for its spawned threads by calling {\tt
  pthread\_join}.
\end{itemize}
\end{frame}

\begin{frame}
\frametitle{Join a Pthread}
\centerline{\pgfimage[height=0.5\textheight]{joining}}
\end{frame}

\begin{frame}
\frametitle{Pthread Attribute} Steps to set threads as
joinable.\footnote{\url{https://computing.llnl.gov/tutorials/pthreads/}}.
\begin{itemize}
\item Declare a pthread attribute variable of type {\tt
  pthread\_attr\_t}.
\item Initialize the attribute variable with {\tt
  pthread\_attr\_init()}.
\item Set the attribute detached status with {\tt
  pthread\_attr\_setdetachstate()}.
\item When done, free library resources used by the attribute with
  {\tt pthread\_attr\_destroy()}.
\end{itemize}
\end{frame}

\begin{frame}
\frametitle{\tt pthread\_attr\_init}
\programlistingoption{pthread-attr-init}{Initialize/destroy attributes}{basicstyle={\small \tt}}
\begin{itemize}
\item Initialize and destroy the attribute of a thread.
\end{itemize}
\end{frame}

\begin{frame}
\frametitle{\tt pthread\_attr\_setdetachstate}
\programlistingoption{pthread-attr-setdetachstate}{Set the attribute}{basicstyle={\footnotesize \tt}}
\begin{itemize}
\item Set the attribute of a thread.
\item In our example we want to set the threads so that they can join
  with the main thread, so we will set {\tt PTHREAD\_CREATE\_JOINABLE}.
\end{itemize}
\end{frame}

\begin{frame}
\frametitle{$n$ Queen with Pthread}
\programlistingoption{queen-pthread-sum-join}{Main program}{linerange=main-declaration_end,basicstyle={\small \tt},emph={attr,pthread_attr_t,pthread_attr_init,pthread_attr_setdetachstate,PTHREAD_CREATE_JOINABLE}}
\end{frame}

\begin{frame}
\frametitle{Steps}
\begin{itemize}
\item Declare a variable {\tt attr} of type {\tt pthread\_attr\_t}
  and initialize it with {\tt pthread\_attr\_init()}.
\item Set the attribute to {\tt PTHREAD\_CREATE\_JOINABLE} with {\tt
  pthread\_attr\_setdetachstate()}.
\item Use the address of {\tt attr} as the second argument in {\tt
  pthread\_create} while creating threads.
\item Destroy {\tt attr}.
\end{itemize}
\end{frame}

\begin{frame}
\frametitle{$n$ Queen with Pthread}
\programlistingoption{queen-pthread-sum-join}{Main program}{linerange=declaration_end-join,basicstyle={\scriptsize \tt},emph={attr,pthread_attr_t,pthread_create,pthread_attr_destroy}}
\end{frame}


\begin{frame}
\frametitle{$n$ Queen with Pthread}
\programlistingoption{queen-pthread-sum-join}{Main program}{linerange=join-end,basicstyle={\scriptsize \tt},emph={threads,pthread_join,numSolution}}
\end{frame}

\begin{frame}
\frametitle{Join a Pthread}
\programlisting{pthread_join}{\tt pthread\_join}
\begin{description}[l]
\item[\tt thread] The thread you want to wait for.  This is the
  variable you used to call {\tt pthread\_create}.
\item[\tt value\_ptr] A pointer to a pointer to the return value from
  the spawned thread.  We do not return anything from the spawned
  thread so we set it to {\tt NULL}.
\end{description}
\end{frame}

\begin{frame}
\frametitle{Demonstration}
\begin{itemize}
\item Run and time the queen-pthread-sum-join.c program.
\end{itemize}
\end{frame}

\begin{frame}
\frametitle{Discussion}
\begin{itemize}
\item Does this program produce correct results?
\item Does this program {\em always} produce correct results?
\end{itemize}
\end{frame}

\begin{frame}
\frametitle{A Global Counter}
\begin{itemize}
\item The previous program uses a recursive function {\tt queen} to
  compute the number of solutions.
\item We will change the program so that {\tt queen} does not return
  the number of solutions it finds -- instead whenever it finds a
  solution it adds 1 to the global counter {\tt numSolution}.
\end{itemize}
\end{frame}

\begin{frame}
\frametitle{}
\programlistingoption{queen-pthread-sum-join-race}{\tt queen}{linerange=queen-go,emph={void,numSolution}}
\end{frame}

\begin{frame}
\frametitle{Changes}
\begin{itemize}
\item The return type of {\tt queen} is changed to {\tt void} since it
  does not return the number of solutions anymore.
\item The program structure is simplified since we do not keep track
  of the number of solutions found.
\end{itemize}
\end{frame}

\begin{frame}
\frametitle{}
\programlistingoption{queen-pthread-sum-join-race}{\tt goQueen}{linerange=go-main}
\end{frame}

\begin{frame}
\frametitle{Demonstration}
\begin{itemize}
\item Run the queen-pthread-sum-join-race.c program.
\end{itemize}
\end{frame}

\begin{frame}
\frametitle{Discussion}
\begin{itemize}
\item Is the answer correct? If not what is the reason?
\item Does the previous program queen-pthread-sum-join.c have the same
  problem?  If yes why it did not show up?
\end{itemize}
\end{frame}

\begin{frame}
\frametitle{Race}
\begin{itemize}
\item In fact the race condition exists in queen-pthread-sum-join.c.
  It doe not show up because there are only a few additions to {\tt
    numSolution}.
\item When we increase the number of additions to {\tt numSolution},
  it becomes much easier for the race condition to occur.
\item The solution is to use the return value of {\tt pthread\_exit} to
  return the number of solutions found in a thread.
\end{itemize}
\end{frame}

\begin{frame}
\frametitle{\tt pthread\_exit}
\programlisting{pthread_exit}{\tt pthread\_exit}
\programlisting{pthread_join}{\tt pthread\_join}
\end{frame}

\begin{frame}
\frametitle{Return value}
\begin{itemize}
\item We will find a place for the spawned thread to store the number
  of solutions it finds, and return the address by {\tt pthread\_exit}
  to the main thread.
\item The main thread will use the second parameter of {\tt
  pthread\_join} to retrieve the number of solutions a thread found.
\end{itemize}
\end{frame}

\begin{frame}
\frametitle{No Global}
\programlistingoption{queen-pthread-sum-join-correct}{Headers}{linerange=begin-ok,basicstyle={\tt},emph={num,pthread_exit}}
\end{frame}

\begin{frame}
\frametitle{No Global}
\begin{itemize}
\item We remove the global variable {\tt numSolution}, which causes
  the race condition.
\end{itemize}
\end{frame}


\begin{frame}
\frametitle{\tt goQueen}
\programlistingoption{queen-pthread-sum-join-correct}{\tt goQueen}{linerange=go-main,basicstyle={\footnotesize \tt},emph={num,pthread_exit}}
\end{frame}

\begin{frame}
\frametitle{Heap}
\begin{itemize}
\item We decide to put the number of solutions a spawned thread found
  in the heap, with the address stored in {\tt num}.
\item Recall that all threads share the heap.
\item This address, i.e., the value of num, is passed back with {\tt
  pthread\_exit}.
\end{itemize}
\end{frame}

\begin{frame}
\frametitle{\tt main}
\programlistingoption{queen-pthread-sum-join-correct}{\tt main}{linerange=join-end,basicstyle={\footnotesize \tt},emph={numSolution,num,pthread_join}}
\end{frame}

\begin{frame}
\frametitle{Return Value}
\begin{itemize}
\item Note that the API requires that we pass the address of a pointer
  into {\tt pthread\_join}.
\item The value returned by {\tt pthread\_join} is then added into
  {\tt numSolution}, which is now a local variable.
\end{itemize}
\end{frame}

\begin{frame}
\frametitle{Demonstration}
\begin{itemize}
\item Run and time the queen-pthread-sum-join-correct.c program.
\end{itemize}
\end{frame}

\begin{frame}
\frametitle{Implementation}
\begin{itemize}
\item Implement a parallel program that solves the $n$ queen problem
  with pthread.  You need to sum up the number of solutions from all
  threads correctly.
\item Measure the speedup and efficiency.
\end{itemize}
\end{frame}

\begin{frame}
\frametitle{Discussion}
\begin{itemize}
\item Does this program produce correct results?
\item Compare the timing from queen-pthread-sum-join-correct.c and
  queen-pthread-sum-join-race.c
\end{itemize}
\end{frame}

\subsection{\tt pthread\_mutex\_lock}

\begin{frame}
\frametitle{Mutex}
\begin{itemize}
\item We can prevent the race condition by allowing only thread to
  access the shared variable {\tt numSolution}.
\item Pthread library provide a {\em mutex} (mutual exclusion)
  mechanism that allows only one thread to proceed in execution.  This
  effectively provide a way to implement a critical section.
\end{itemize}
\end{frame}

\begin{frame}
\frametitle{Steps}
\begin{itemize}
\item Declare a mutex variable of type {\tt pthread\_mutex\_t}.  This
  variable must be global because every thread will use it.
\item Initialize the mutex with {\tt pthread\_mutex\_init}.
\item Lock the mutex with {\tt pthread\_mutex\_lock}.
\item Unlock the mutex with {\tt pthread\_mutex\_unlock}.
\end{itemize}
\end{frame}

\begin{frame}
\frametitle{Declaration}
\programlistingoption{queen-pthread-sum-join-race-mutex}{Declaration}{linerange=declaration-ok,basicstyle={\small \tt},emph={pthread_mutex_t}}
\end{frame}

\begin{frame}
\frametitle{Mutex Variable}
\begin{itemize}
\item Declare a global mutex variable {\tt numSolutionLock}.
\item We will use this mutex variable to synchronize the access to the
  global counter {\tt numSolution}.
\end{itemize}
\end{frame}

\begin{frame}
\frametitle{\tt main}
\programlistingoption{queen-pthread-sum-join-race-mutex}{\tt queen}{linerange=main-create,basicstyle={\small \tt},emph={pthread_mutex_init,pthread_mutex_t,numSolutionLock}}
\end{frame}

\begin{frame}
\frametitle{\tt pthread\_mutex\_init}
\programlistingoption{pthread-mutex-init}{Initialize a mutex}{basicstyle={\scriptsize \tt}}
\begin{description}[l]
\item[\tt mutex] The mutex to initialize.
\item[\tt attr] The attribute of the mutex to set.  Set to {\tt NULL}
  for the default.
\end{description}
\end{frame}

\begin{frame}
\frametitle{\tt main}
\begin{itemize}
\item Initialize {\tt numSolutionLock} with {\tt pthread\_mutex\_init}.
\item Use {\tt NULL} for the default mutex attribute value.
\end{itemize}
\end{frame}

\begin{frame}
\frametitle{\tt pthread\_mutex\_lock}
\programlistingoption{pthread-mutex-lock}{Lock/Unlock a mutex}{basicstyle={\small \tt}}
\begin{description}[l]
\item[\tt mutex] The mutex to lock/unlock.
\end{description}
\end{frame}

\begin{frame}
\frametitle{\tt queen}
\programlistingoption{queen-pthread-sum-join-race-mutex}{\tt queen}{linerange=queen-go,basicstyle={\small \tt},emph={pthread_mutex_lock,pthread_mutex_unlock,numSolutionLock,numSolution}}
\end{frame}

\begin{frame}
\frametitle{\tt queen}
\begin{itemize}
\item Whenever the function {\tt queen} finds a solution, it needs to
  lock {\tt numSolution} with mutex {\tt numSolutionLock} by calling
  {\tt pthread\_mutex\_lock}.
\item If it cannot acquire the lock it will have to wait.
\item After adding 1 to {\tt numSolution}, {\tt queen} releases the
  lock by calling {\tt pthread\_mutex\_unlock}.
\end{itemize}
\end{frame}

\begin{frame}
\frametitle{\tt main}
\programlistingoption{queen-pthread-sum-join-race-mutex}{\tt queen}{linerange=join-end,basicstyle={\small \tt},emph={pthread_mutex_destroy,numSolutionLock}}
\end{frame}

\begin{frame}
\frametitle{\tt pthread\_mutex\_destroy}
\programlistingoption{pthread-mutex-destroy}{\tt pthread\_destroy}{basicstyle={\footnotesize \tt}}
\end{frame}

\begin{frame}
\frametitle{\tt main}
\begin{itemize}
\item After all spawned threads finish, we release the mutex by calling
{\tt pthread\_mutex\_destroy}.
\end{itemize}
\end{frame}

\begin{frame}
\frametitle{Demonstration}
\begin{itemize}
\item Run and time the queen-pthread-sum-join-race-mutex.c program.
\end{itemize}
\end{frame}

\begin{frame}
\frametitle{Discussion}
\begin{itemize}
\item Does this program produce correct results?
\item Compare the timing from queen-pthread-sum-join-race-mutex.c with
  that from queen-pthread-sum-join-correct.c.  
\end{itemize}
\end{frame}

\end{document}

