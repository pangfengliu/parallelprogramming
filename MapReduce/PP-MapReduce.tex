\documentclass{beamer}
\usetheme{Warsaw}

\usepackage{color}
\usepackage{listings}
\usepackage{url}
\usepackage{pgf}

% to specify the size of fonts in lst environment

\lstset{showstringspaces=false}
\lstset{language=C}
\lstset{basicstyle=\tt}
\lstset{keywordstyle=\color{blue}}
\lstset{emphstyle=\color{red}}
\lstset{numbers=left}
\lstset{frame=single}
\lstset{rangeprefix=\/*\ }
\lstset{rangesuffix=\ *\/}
\lstset{includerangemarker=false}

%% Program listing macro for the first page with default setting.  

\newcommand{\programlisting}[2]
{\renewcommand{\lstlistingname}{Example}
\lstinputlisting[caption={#2}]{#1.c}}

\newcommand{\programlistingoption}[3]
{\renewcommand{\lstlistingname}{Example}
\lstinputlisting[caption={(#1.c) #2},#3]{#1.c}}

\newcommand{\executionlisting}[2]
{{\renewcommand{\lstlistingname}{Execution}
\lstinputlisting[caption={}]{#1.out}}}

\newcommand{\executionlistingoption}[3]
{{\renewcommand{\lstlistingname}{Execution}
\lstinputlisting[caption={},#3]{#1.out}}}

\newcommand{\filelisting}[2]
{\renewcommand{\lstlistingname}{Example}
\lstinputlisting[caption={#2}]{#1}}

\newcommand{\prototype}[2]
{{\renewcommand{\lstlistingname}{Prototype}
\lstinputlisting[caption={#2}]{#1.h}}}

\newcommand{\pragma}[2]
{{\renewcommand{\lstlistingname}{Pragma}
\lstinputlisting[caption={#2}]{#1.h}}}


\begin{document}

\title{MapReduce Programming}

\author{Pangfeng Liu \\ National Taiwan University}

\begin{frame}
  \titlepage
\end{frame}

\section{Introduction}

\begin{frame}
  \frametitle{Topics}
  \begin{itemize}
    \item Google File System
    \item MapReduce Programming Model
  \end{itemize}
\end{frame}

\section{Google File System}
\begin{frame}
  \frametitle{GFS in a Nutshell}
  \begin{itemize}
    \item Google File System (GFS) is a scalable distributed file
      system for large distributed data-intensive applications.
    \item GFS provides fault tolerance while running on inexpensive
      commodity hardware, and it delivers high aggregate performance
      to a large number of clients.
  \end{itemize}
\end{frame}

\begin{frame}
  \frametitle{Objectives} 
  \begin{itemize}
  \item GFS shares many of the same goals as previous distributed file
    systems such as performance, scalability, reliability, and
    availability. However, its design has been driven by key
    observations of our application workloads and technological
    environment.
  \end{itemize}
\end{frame}

\begin{frame}
  \frametitle{Observations}
  \begin{itemize}
    \item Component failures are the norm rather than the exception.
    \item Files are huge by traditional standards.
    \item Most files are mutated by appending new data rather than
      overwriting existing data. Random writes within a file are
      practically non-existent.
    \item Co-designing the applications and the file system API
      benefits the overall system by increasing our flexibility.
  \end{itemize}
\end{frame}

\begin{frame}
  \frametitle{Assumptions}
  \begin{itemize}
    \item The system is built from many inexpensive commodity
      components that often fail.
    \item The system stores a modest number of large files.
    \item The workloads primarily consist of two kinds of reads: large
      streaming reads and small random reads.
    \item The workloads also have many large, sequential writes that
      append data to files.
    \item The system must efficiently implement well-defined semantics
      for multiple clients that concurrently append to the same file.
    \item High sustained bandwidth is more important than low latency.
  \end{itemize}
\end{frame}

\begin{frame}
  \frametitle{Characteristics}
  \begin{itemize}
    \item One master stores metadata including data locations and
      multiple chunk servers stores actual data.
    \item Metadata are stored in memory for fast access.
    \item Actual data has three replicas for fault tolerance, and
      replicas are stored within and not within locality.
    \item Complicated protocol for storing data.
  \end{itemize}
\end{frame}

\begin{frame}
  \frametitle{Discussion}
  \begin{itemize}
    \item Descirbe the advantage and disadvantages of large block size
      in a file system.
  \end{itemize}
\end{frame}

\section{MapReduce Programming}

\begin{frame}
  \frametitle{MapReduce}
  \begin{itemize}
  \item A programming model and an associated implementation for
    processing and generating large data sets.
  \item Programs written in MapReduce style are automatically
    parallelized and executed on a large cluster of commodity
    machines.
  \item Google implementation of MapReduce runs on a large cluster of
    commodity machines and is highly scalable.
  \end{itemize}
\end{frame}

\subsection{Introduction}

\begin{frame}
  \frametitle{Simple Computation}
  \begin{itemize}
  \item Google has implemented hundreds of special-purpose
    computations that process large amounts of raw data.
  \item Most such computations are conceptually straightforward but
    the input data is usually large and the computations have to be
    distributed across hundreds or thousands of machines.
  \item The issues of how to parallelize the computation, distribute
    the data, and handle failures obscure the original simple
    computation with large amounts of complex code to deal with these
    issues.
  \end{itemize}
\end{frame}

\begin{frame}
  \frametitle{Abstraction}
  \begin{itemize}
    \item Google designed a new abstraction that allows programmers to
      express the simple computations but hides the messy details of
      parallelization, fault-tolerance, data distribution and load
      balancing.  
    \item The programmers focus on the computation, not the
      parallelization.
  \end{itemize}
\end{frame}

\begin{frame}
  \frametitle{Abstraction}
  \begin{itemize}
    \item The abstraction applies a {\em map} operation to each
      logical record in our input in order to compute a set of
      intermediate {\em key/value pairs}.
    \item Then the system applies a {\em reduce} operation to all the
      values that shared the {\em same} key, in order to combine the
      derived data.
  \end{itemize}
\end{frame}

\begin{frame}
  \frametitle{Discussion}
  \begin{itemize}
    \item Give example of most high level and most low level of
      parallel programming you are aware of.
    \item Discuss the advantage and disadvantage of abstraction on
      parallel programming.
  \end{itemize}
\end{frame}

\subsection{Programming Model}

\begin{frame}
  \frametitle{Map and Reduce}
  \begin{itemize}
  \item The programming model consists of a {\em map} function, a {\em
    reduce} function, and a {\em shuffle stage}.
  \item Map and reduce are borrowed from functional programming.
  \end{itemize}
\end{frame}

\begin{frame}
  \frametitle{Stages}
  \begin{itemize}
    \item In the map stage the map function reads inputs and produces
      intermediate key/value pairs.
    \item In the shuffle stage the system processes intermediate
      key/value pairs so that later the reduce function can read them.
    \item In the reduce stage the reduce function reads the processed
      intermediate key/value pairs and produces the output.
  \end{itemize}
\end{frame}

\begin{frame}
  \frametitle{Responsibility}
  \begin{itemize}
    \item The programmer provides the map and reduce function.
    \item The MapReduce runtime system does the shuffling.
  \end{itemize}
\end{frame}

\begin{frame}
  \frametitle{Discussion}
  \begin{itemize}
    \item What does a programmer have to do in MapReduce style of
      programming?
  \end{itemize}
\end{frame}

\subsubsection{Map}

\begin{frame}
  \frametitle{Map}
  \begin{itemize}
    \item Map is a function.
  \end{itemize}
  \begin{description}
  \item[input] An input I.
  \item[output] A multi-set $\{ (k_1, v_1), \ldots , (k_n, v_n) \}$.
  \end{description}
\end{frame}

\begin{frame}
  \frametitle{A Map Example} 
  \begin{itemize}
    \item Given a string, produces a multi-set of pairs that are in
      the form of $(w, 1)$, where $w$ is a word in the string.
    \item The input is ``this is a book not a pencil''.
    \item The output will be (this, 1), (is, 1), (a, 1), (book,
      1), (not, 1), (a, 1), (pencil, 1).
  \end{itemize}
\end{frame}

\begin{frame}
  \frametitle{An Example} 
  \begin{itemize}
    \item The same map function may have different inputs.
    \item Assume that the input is ``this is a pencil not a chair''.
    \item the output will be (this, 1), (is, 1), (a, 1), (pencil,
      1), (not, 1), (a, 1), (chair, 1).
  \end{itemize}
\end{frame}

\begin{frame}
  \frametitle{Key/Value Pairs}
  \begin{itemize}
    \item Now the system collects all the output from all inputs from
      map function.  This is called {\em intermediate key/value
        pairs}.
    \item From the previous example we have the intermediate key/value
      pairs from two inputs as (this, 1), (is, 1), (a, 1), (book,
      1), (not, 1), (a, 1), (pencil, 1), (this, 1), (is, 1), (a, 1),
      (pencil, 1), (not, 1), (a, 1), (chair, 1).
  \end{itemize}
\end{frame}


\begin{frame}
  \frametitle{Another Example}
  \begin{itemize}
    \item Given a document $d$, produces a set of pairs that ares in
      the form of $(w, d)$, where $w$ is a word in $u$.
      \item The input is a document ``readme'', which has words
        ``this'', ``is'', ``a'', ``pencil''.
      \item The output will be a set (this, readme), (is,
        readme), (a, readme), (pencil, readme).
  \end{itemize}
\end{frame}

\begin{frame}
  \frametitle{An Example} 
  \begin{itemize}
    \item The same map function may have different inputs.
    \item The input is a document ``note'', which has words ``that'',
      ``is'', ``a'', ``chair''.
    \item The output will be (that, note), (is, note), (a,
      note), (chair, note).
  \end{itemize}
\end{frame}

\begin{frame}
  \frametitle{Key/Value Pairs}
  \begin{itemize}
    \item We have the intermediate key/value pairs as (this,
      readme), (is, readme), (a, readme), (pencil, readme), (that,
      note), (is, note), (a, note), (chair, note).
  \end{itemize}
\end{frame}

\begin{frame}
  \frametitle{Discussion}
  \begin{itemize}
    \item Can you guess the purpose of these two examples?  What kind
      of problem they want to solve?
  \end{itemize}
\end{frame}

\subsubsection{Shuffle}

\begin{frame}
  \frametitle{Shuffle}
  \begin{itemize}
    \item The {\em shuffle} in MapReduce means transferring data from
      map to reduce via network transfer.
    \item Before we pass the intermediate key/value pairs to the
      reduce function, we must {\em sort} them.
    \item The reason is that a reduce function will work on the values
      from the {\em same} key only, so we need to shuffle the pairs so
      that the pairs of the same key go to the same reduce function.
  \end{itemize}
\end{frame}

\begin{frame}
  \frametitle{Key/Value Pairs}
  \begin{itemize}
    \item From the previous example we have the intermediate key/value
      pairs as (this, 1), (is, 1), (a, 1), (book, 1), (not, 1),
      (a, 1), (pencil, 1), (this, 1), (is, 1), (a, 1), (pencil, 1),
      (not, 1), (a, 1), (chair, 1)
  \end{itemize}
\end{frame}

\begin{frame}
  \frametitle{Shuffle}
  \begin{itemize}
    \item The intermediate key value pairs will be sorted according to
      keys for later use.
    \item We will have (a, 1), (a, 1), (a, 1), (a, 1), (book, 1),
      (chair, 1) (is, 1), (is, 1), (not, 1), (not, 1), (pencil, 1),
      (pencil, 1), (this, 1), (this, 1)
  \end{itemize}
\end{frame}

\begin{frame}
  \frametitle{Group Values Together}
  \begin{itemize}
    \item The values of the same key will be grouped into a list
    \item We will have the following lists.
      \begin{itemize}
        \item (a, (1, 1, 1, 1))
        \item (book, (1))
        \item (chair, (1))
        \item (is, (1, 1))
        \item (not, (1, 1))
        \item (pencil, (1, 1))
        \item (this, (1, 1))
      \end{itemize}
  \end{itemize}
\end{frame}

\begin{frame}
  \frametitle{Discussion}
  \begin{itemize}
    \item Can you guess the purpose of this example now? What kind of
      problem they want to solve?
    \item Now if we want to solve this problem, what else do we (or
      the reducer) need to do?
  \end{itemize}
\end{frame}

\begin{frame}
  \frametitle{Key/Value Pairs}
  \begin{itemize}
    \item We have the intermediate key/value pairs as (this,
      readme), (is, readme), (a, readme), (pencil, readme), (that,
      note), (is, note), (a, note), (chair, note).
  \end{itemize}
\end{frame}

\begin{frame}
  \frametitle{Shuffle}
  \begin{itemize}
    \item The intermediate key value pairs will be sorted according to
      keys for later use.
    \item We will have (a, note), (a, readme), (chair, note), (is,
      readme), (is, note), (pencil, readme), (that, note), (this,
      readme).
  \end{itemize}
\end{frame}

\begin{frame}
  \frametitle{Group Values Together}
  \begin{itemize}
    \item The values of the same key will be grouped into a list
    \item We will have the following lists.
      \begin{itemize}
        \item (a, (note, readme))
        \item (chair, (note))
        \item (is, (note, readme))
        \item (pencil, (readme))
        \item (that, (note))
        \item (this, (readme))
      \end{itemize}
  \end{itemize}
\end{frame}

\begin{frame}
  \frametitle{Discussion}
  \begin{itemize}
    \item Can you guess the purpose of this example now? What kind of
      problem they want to solve?
    \item Now if we want to solve this problem, what else do we (or
      the reducer) need to do?
  \end{itemize}
\end{frame}

\subsubsection{Reduce}

\begin{frame}
  \frametitle{Reduce}
  \begin{itemize}
    \item Reduce is a function.
  \end{itemize}
  \begin{description}
  \item[input] A key and a list of values.
  \item[output] A result from the list of values.
  \end{description}
\end{frame}

\begin{frame}
  \frametitle{A Reduce Example} 
  \begin{itemize}
    \item Given a key and a list of numbers, produces a pair of the
      key and the sum of these numbers.
  \end{itemize}
\end{frame}

\begin{frame}
  \frametitle{Word Count} 
  \begin{itemize}
    \item Given (a, (1, 1, 1, 1)) produces (a, 4).
    \item Given (book, (1)) produces (book, 1).
    \item Given (chair, (1)) produces (chair, 1).
    \item Given (is, (1, 1)) produces (is, 2).
    \item Given (not, (1, 1)) produces (not, 2).
    \item Given (pencil, (1, 1)) produces (pencil, 2).
    \item Given (this, (1, 1)) produces (this, 2).
  \end{itemize}
\end{frame}

\begin{frame}
  \huge{We have a word counting program!!!}
\end{frame}

\begin{frame}
  \frametitle{Discussion}
  \begin{itemize}
    \item Can you think of any optimization that a map can do to
      reduce the amount of data from map to reduce in this example?
  \end{itemize}
\end{frame}

\begin{frame}
  \frametitle{A Reduce Example} 
  \begin{itemize}
    \item Given a key and a list of values, produces the {\em same
      thing}.
    \item Like an identify function.
  \end{itemize}
\end{frame}

\begin{frame}
  \frametitle{Word Count} 
  \begin{itemize}
  \item Given (a, (note, readme)) produces (a, (note, readme)).
  \item Given (chair, (note)) produces (chair, (note)).
  \item Given (is, (note, readme)) produces (is, (note, readme)).
  \item Given (pencil, (readme)) produces (pencil, (readme)).
  \item Given (that, (note)) produces (that, (note)).
  \item Given (this, (readme)) produces (this, (readme)).
  \end{itemize}
\end{frame}

\begin{frame}
  \huge{We have a word index program!!!}
\end{frame}

\begin{frame}
  \frametitle{Discussion}
  \begin{itemize}
    \item If the map function produces a multi-set instead of a set,
      then what will be the output?
  \end{itemize}
\end{frame}

\begin{frame}
  \frametitle{Mapper and Reducer}
  \begin{itemize}
    \item Both map and reduce functions are run by multiple machines
      in parallel.
    \item A machine running a map function is called a {\em mapper}.
    \item A machine running a reduce function is called a {\em
      reducer}.
    \item We will assume that we have $M$ mappers and $R$ reducers.
  \end{itemize}
\end{frame}

\begin{frame}
  \frametitle{Partition Function}
  \begin{itemize}
    \item A {\em partition function} determines the reducer an
      intermediate key/value pair will go to.
    \item Usually we use a simple hash function to hash the key into a
      integer from 0 to $R - 1$.
    \item Each reducer writes its data according to its index (from 0
      to $R -1$.
    \item The final answer is a concatenation of all files according
      to reducer ids.
  \end{itemize}
\end{frame}

\begin{frame}
  \frametitle{Discussion}
  \begin{itemize}
  \item Can any reducer run before any mapper finishes?  Explain your
    answer.
  \item How do we do sorting with MapReduce if we know the key
    distribution?  In particular, how do we design our partition
    function, so that the final result is sorted?
  \end{itemize}
\end{frame}

\subsubsection{Example Code}

\begin{frame}
  \frametitle{Map}
  \begin{itemize}
    \item The user defined map function extends (inherits) the base
      class {\tt MapReduceBase}
    \item The first two template parameters of interface {\tt Mapper}
      describe the input, and second two template parameters describe
      the output.
    \item The output type is a template using the parameter of {\tt
      Mapper}, and is the type of intermediate key/value pairs.
  \end{itemize}
\end{frame}

\begin{frame}
  \frametitle{Map Parameters}
  \begin{description}
    \item[key] The line number of file.
    \item[value] The line at {\tt key}.
    \item[output] The output object.
    \item[reporter] The error/information handle.
  \end{description}
\end{frame}

\begin{frame}
  \frametitle{\tt StringTokenizer}
  \begin{itemize}
  \item We convert value to string, then convert the string with the
    constructor of {\tt StringTokenizer}.
  \item The {\em hasMoreToken} and {\em nextToken} are methods of {\tt
    StringTokenizer} that works as an iterator.
  \end{itemize}
\end{frame}

\begin{frame}
  \frametitle{\tt OutputColelctor}
  \begin{itemize}
  \item {\tt OutputColelctor} is the type of intermediate key value.
  \item We write the string (of type {\tt Text}) first, then a constant
    1.
  \end{itemize}
\end{frame}

\begin{frame}
  \frametitle{Map}
  \programlistingfirst{WordCount.java}{}{\scriptsize}{mapper}{mapperend}
\end{frame}

\begin{frame}
  \frametitle{Discussion}
  \begin{itemize}
    \item Did we use the parameter {\tt key}?
    \item Guess why the parameter {\tt key} has type {\tt
      LongWritable} not simply a {\tt IntWritable}.
  \end{itemize}
\end{frame}

\begin{frame}
  \frametitle{Reduce}
  \begin{itemize}
    \item The user defined {\tt reduce} function extends (inherits)
      the base class {\tt MapReduceBase}
    \item The first two template parameters of interface {\tt Reducer}
      describe the input, and second two template parameters describe
      the output.
    \item The input type is a template using the parameter of {\tt
      Reducer}, and is the type of intermediate key/value pairs.

  \end{itemize}
\end{frame}

\begin{frame}
  \frametitle{Reduce Parameters}  \begin{description}
    \item[key] The intermediate key.
    \item[values] intermediate values.
    \item[output] The output object.
    \item[reporter] The error/information handle.
  \end{description}
  Note that we have only one key but multiple values, so we need an
  iterator over values.
\end{frame}

\begin{frame}
  \frametitle{\tt IntWritable}
  \begin{itemize}
  \item The {\em hasNext} and {\em next} are methods of {\tt Iterator}
    that works as an iterator over {\tt IntWritable}.
  \item {\tt get} is a method to get the actual value of an {\tt
    IntWritable}.
  \end{itemize}
\end{frame}

\begin{frame}
  \frametitle{\tt OutputColelctor}
  \begin{itemize}
  \item {\tt OutputColelctor} is the type of intermediate key value.
  \item We write the string (of type {\tt Text}) first, then the sum.
  \end{itemize}
\end{frame}

\begin{frame}
  \frametitle{Reduce}
  \programlistingfirst{WordCount.java}{}{\scriptsize}{mapperend}{reducerend}
\end{frame}

\begin{frame}
  \frametitle{Discussion}
  \begin{itemize}
    \item Check the consistency of the second two parameter of {\tt
      Map} and first two parameter of {\tt Reduce}.  Why do they have
      to be the same?
  \end{itemize}
\end{frame}

\begin{frame}
  \frametitle{Reduce}
  \programlistingfirst{WordCount.java}{}{\scriptsize}{mapperend}{reducerend}
\end{frame}

\begin{frame}
  \frametitle{Main Program}
  \begin{itemize}
    \item The {\tt JobConf} is the job configuration object.
    \item First we set the name of the job, which has to be the same
      as our object name (and the file name).
    \item Then we set the mapper and reducer. 
    \item We also set the combiner, which is also the reducer.
  \end{itemize}
\end{frame}

\begin{frame}
  \frametitle{Main Program}
  \programlistingfirst{WordCount.java}{}{\scriptsize}{reducerend}{setkeyvalue}
\end{frame}

\begin{frame}
  \frametitle{Set Key/Value Type}
  \begin{itemize}
    \item We set the output type for mapper and reducer.
    \item The API for mapper is {\tt setMapOutputKeyClass} and {\tt
      setMapOutputValueClass}
    \item The API for reducer is {\tt setOutputKeyClass} and {\tt
      setOutputValueClass}
  \end{itemize}
\end{frame}


\begin{frame}
  \frametitle{Set Key/Value Type}
  \programlistingfirst{WordCount.java}{}{\scriptsize}{setkeyvalue}{setio}
\end{frame}

\begin{frame}
  \frametitle{Set I/O Type}
  \begin{itemize}
    \item We set the input and output for the main program.
    \item {\tt setInputFormat} is for mappers to read.
    \item {\tt setOutputFormat} is for reducers to write.
    \item {\tt setInputPaths} is the path name of the input
    \item {\tt setOutputPaths} is the path name of the output.
  \end{itemize}
\end{frame}

\begin{frame}
  \frametitle{Reduce}
  \programlistingfirst{WordCount.java}{}{\scriptsize}{setio}{mainend}
\end{frame}

\begin{frame}
  \frametitle{Discussion}
  \begin{itemize}
    \item Why the combiner is also the reducer?
  \end{itemize}
\end{frame}

\subsection{Implementation}

\begin{frame}
  \frametitle{Implementation}
  \begin{itemize}
    \item Commodity machines and networking hardware are used.
    \item A cluster consists of hundreds or thousands of machines,
      and therefore machine failures are common.  
    \item Storage is provided by inexpensive disks attached directly
      to individual machines.
    \item GFS is used to manage the data stored on these disks. The
      file system uses replication to provide availability and
      reliability on top of unreliable hardware.
  \end{itemize}
\end{frame}

\begin{frame}
  \frametitle{Stages}
  \begin{itemize}
    \item The execution of a MapReduce program goes through seven
      stages.
      \begin{itemize}
      \item Split input
      \item Master initialization
      \item Start mappers
      \item Network transfer
      \item Start reducers
      \item Reduces output
      \item Master cleanup
      \end{itemize}
  \end{itemize}
\end{frame}

\begin{frame}
  \frametitle{Split Input}
  \begin{itemize}
    \item The MapReduce library in the user program first splits the
      input files into $M$ pieces of typically 16 megabytes to 64
      megabytes (MB) per piece.
    \item The MapReduce library then starts up many copies of the
      program on a cluster of machines.
  \end{itemize}
\end{frame}

\begin{frame}
\frametitle{MapReduce}
\centerline{\pgfimage[height=0.7\textheight]{MapReduce.png}}
\end{frame}

\begin{frame}
  \frametitle{Discussion}
  \begin{itemize}
    \item Why input files are split into $M$ pieces of typically 16
      megabytes to 64 megabytes (MB) per piece?
  \end{itemize}
\end{frame}

\begin{frame}
  \frametitle{Master Initialization}
  \begin{itemize}
    \item One of the copies of the program is the master and the rest
      are workers that are assigned work by the master. 
    \item There are $M$ map tasks and $R$ reduce tasks to assign.
    \item The master picks idle workers and assigns each one a map
      task or a reduce task.
  \end{itemize}
\end{frame}
    
\begin{frame}
  \frametitle{Start Mappers}
  \begin{itemize}
    \item A worker who is assigned a map task reads the contents of
      the corresponding input split. 
    \item It parses key/value pairs out of the input data and passes
      each pair to the user-defined Map function. 
    \item The intermediate key/value pairs produced by the Map
      function are buffered in {\em memory}.
  \end{itemize}
\end{frame}

\begin{frame}
\frametitle{MapReduce}
\centerline{\pgfimage[height=0.7\textheight]{MapReduce.png}}
\end{frame}

\begin{frame}
  \frametitle{Network Transfer}
  \begin{itemize}
    \item Periodically, the buffered pairs are written to local disk,
      partitioned into R regions by the partitioning function. 
    \item The locations of these buffered pairs on the local disk are
      passed back to the master, who is responsible for forwarding
      these locations to the reduce workers.
  \end{itemize}
\end{frame}

\begin{frame}
\frametitle{MapReduce}
\centerline{\pgfimage[height=0.7\textheight]{MapReduce.png}}
\end{frame}

\begin{frame}
  \frametitle{Start Reducers}
  \begin{itemize}
    \item When a reduce worker is notified by the master about these
      locations, it uses remote procedure calls to read the buffered
      data from the local disks of the map workers. 
    \item When a reduce worker has read {\em all} intermediate data,
      it sorts it by the intermediate keys so that all occurrences of
      the same key are grouped together.
    \item The sorting is needed because typically many different keys
      map to the same reduce task.
  \end{itemize}
\end{frame}

\begin{frame}
\frametitle{MapReduce}
\centerline{\pgfimage[height=0.7\textheight]{MapReduce.png}}
\end{frame}

\begin{frame}
  \frametitle{Discussion}
  \begin{itemize}
    \item What is remote procedure call?  What requirement is needed
      for doing RPC?
  \end{itemize}
\end{frame}

\begin{frame}
  \frametitle{Reduces Output}
  \begin{itemize}
    \item The reduce worker iterates over the sorted intermediate data
      and for each unique intermediate key encountered, it passes the
      key and the corresponding set of intermediate values to the
      user’s Reduce function. 
    \item The output of the Reduce function is {\em appended} to a
      final output file for this reduce partition.
  \end{itemize}
\end{frame}

\begin{frame}
\frametitle{MapReduce}
\centerline{\pgfimage[height=0.7\textheight]{MapReduce.png}}
\end{frame}

\begin{frame}
  \frametitle{Master Cleanup}
  \begin{itemize}
    \item When all map tasks and reduce tasks have been completed, the
      master wakes up the user program.  
    \item At this point, the MapReduce call in the user program
      returns back to the user code.
  \end{itemize}
\end{frame}

\subsubsection{Fault Tolerance}

\begin{frame}
  \frametitle{Fault Tolerance}
  \begin{itemize}
    \item Since the MapReduce library is designed to help process very
      large amounts of data using hundreds or thousands of machines,
      the library must tolerate machine failures gracefully.
    \item We will discuss worker failure and master failure
  \end{itemize}
\end{frame}

\begin{frame}
  \frametitle{Worker Failure}
  \begin{itemize}
    \item The master pings every worker periodically. If no response
      is received from a worker in a certain amount of time, the
      master marks the worker as failed.
    \item Completed map tasks are re-executed on a failure because
      their output is stored on the local disk(s) of the failed
      machine and is therefore inaccessible.
    \item Completed reduce tasks do not need to be re-executed since
      their output is stored in a global file system.
  \end{itemize}
\end{frame}

\begin{frame}
  \frametitle{Master Failure}
  \begin{itemize}
    \item The current implementation aborts the MapReduce computation
      if the master fails.
  \end{itemize}
\end{frame}

\begin{frame}
  \frametitle{Discussion}
  \begin{itemize}
    \item Why do we re-execute a failed mapper, but not a reducer?
    \item Why master failure is different from worker failure? Explain
      you answer.
  \end{itemize}
\end{frame}

\subsubsection{Locality}

\begin{frame}
  \frametitle{Locality}
  \begin{itemize}
    \item We conserve network bandwidth by taking advantage of the
      fact that the input data, managed by GFS, is stored on the local
      disks of the machines that make up our cluster.
    \item GFS divides each file into 64 MB blocks, and stores several
      copies of each block (typically 3 copies) on different
      machines. 
  \end{itemize}
\end{frame}

\begin{frame}
  \frametitle{Locality}
  \begin{itemize}
    \item The MapReduce master takes the location information of the
      input files into account and attempts to schedule a map task on
      a machine that contains a replica of the corresponding input
      data.
    \item Failing that, it attempts to schedule a map task near a
      replica of that task’s input data (e.g., on a worker machine
      that is on the same network switch as the machine containing the
      data).
  \end{itemize}
\end{frame}

\begin{frame}
  \frametitle{Discussion}
  \begin{itemize}
    \item How dose the system know where to find the replica of a
      block? Who has that location information?
  \end{itemize}
\end{frame}

\subsubsection{Granularity}

\begin{frame}
  \frametitle{Granularity}
  \begin{itemize}
    \item We tend to choose $M$ so that each individual task is
      roughly 16 MB to 64 MB of input data (so that the locality
      optimization described above is most effective), 
    \item We make $R$ a small multiple of the number of worker
      machines we expect to use.
  \end{itemize}
\end{frame}

\begin{frame}
  \frametitle{Straggler}
  \begin{itemize}
    \item A ``straggler'' is a machine that takes an unusually long
      time to complete one of the last few map or reduce tasks in the
      computation.
    \item For example, a machine with a bad disk may experience
      frequent correctable errors that slow its read performance from
      30 MB/s to 1 MB/s.
  \end{itemize}
\end{frame}

\begin{frame}
  \frametitle{Backup Tasks}
  \begin{itemize}
    \item When a MapReduce operation is close to completion, the
      master schedules {\em backup executions} of the remaining
      in-progress tasks.
    \item The task is marked as completed whenever either the primary
      or the backup execution completes.
    \item The sort program takes 44\% longer to complete when the
      backup task mechanism is disabled.
  \end{itemize}
\end{frame}

\begin{frame}
  \frametitle{Discussion}
  \begin{itemize}
    \item If the probability for a machine to be a straggler is $p$,
      what is the pro-ability that a system of 1800 machines does {\em
        not} have a straggler?  Let $p$ be 0.0001 and calculate the
      answer.
  \end{itemize}
\end{frame}
  
\subsection{Optimization}

\begin{frame}
  \frametitle{Partition Function}
  \begin{itemize}
    \item The default partitioning function uses hashing
      (e.g. hash(key) mod R) and tends to result in fairly
      well-balanced partitions.
    \item Sometimes the output keys are URLs, and we want all entries
      for a single host to end up in the same output file. 
    \item We can specify a partitioning function using
      hash(Hostname(urlkey)) mod $R$, and all URLs from the same host
      will end up in the same output file.
  \end{itemize}
\end{frame}

\begin{frame}
  \frametitle{Combiner Function}
  \begin{itemize}
    \item There is significant repetition in the intermediate keys
      produced by each map task, and the user-specified Reduce
      function is commutative and associative.
    \item For example in work counting each map task will produce
      hundreds or thousands of records of the form (the, 1).
    \item We allow the user to specify an optional {\em Combiner
      function} that does partial merging of this data before it is
      sent over the network.
  \end{itemize}
\end{frame}

\begin{frame}
  \frametitle{Example}
  \begin{itemize}
    \item For example, the map function of word counting example after
      seeing ``this is a book not a pencil'', will report (a, (2)),
      instead of (a (1, 1)).
    \item Similar, the map function of word counting example after
      seeing ``this is a pencil not a chair'' will also report (a,
      (2)), instead of (a (1, 1)).
    \item The reducer will see two (a, (2)), instead of two (a, (1,
      1)).
      \item This is like ``preprocessing'' in order to reduce the
        network traffic.
  \end{itemize}
\end{frame}

\begin{frame}
  \frametitle{Discussion}
  \begin{itemize}
    \item Give another example of combiner.
  \end{itemize}
\end{frame}

\begin{frame}
  \frametitle{Local Execution}
  \begin{itemize}
    \item Debugging problems in Map or Reduce functions can be tricky,
      since the actual computation happens in a distributed system of
      several thousand machines.
    \item To help facilitate debugging, profiling, and small-scale
      testing, we have developed an alternative implementation of
      the MapReduce library that sequentially executes all of the work
      for a MapReduce operation on the local machine.
  \end{itemize}
\end{frame}

\begin{frame}
  \frametitle{Counters}
  \begin{itemize}
    \item The MapReduce library provides a counter facility to count
      occurrences of various events.
    \item Some counter values are automatically maintained by the
      MapReduce library, such as the number of input key/value pairs
      processed and the number of output key/value pairs produced.
    \item Users have found the counter facility useful for sanity
      checking the behavior of MapReduce operations.
  \end{itemize}
\end{frame}

\subsection{Performance}

\begin{frame}
  \frametitle{Performance} 
  \begin{itemize} 
    \item The experiments run on a cluster of 1800 machines connected
      by a two-level tree-shaped switched network with approximately
      100-200 Gbps of aggregate bandwidth available at the root.
    \item Each machine had two 2GHz Intel Xeon processors with
        Hyper-Threading enabled, 4GB of memory, two 160GB IDE disks.
    \item The benchmarks include Grep and sort.
  \end{itemize}
\end{frame}

\begin{frame}
\frametitle{Mapper Reading}
\centerline{\pgfimage[height=0.7\textheight]{mapper-read.png}}
\end{frame}

\begin{frame}
  \frametitle{Mapper Reading Rate}
  \begin{itemize}
    \item This graph shows the rate at which input is read. 
    \item The rate peaks at about 13 GB/s and dies off fairly quickly
      since all map tasks finish before 200 seconds have elapsed.
  \end{itemize}
\end{frame}

\begin{frame}
\frametitle{Shuffle}
\centerline{\pgfimage[height=0.7\textheight]{shuffle.png}}
\end{frame}

\begin{frame}
  \frametitle{Network Transfer Rate}
  \begin{itemize}
    \item This graph shows the rate at which data is sent over the
      network from the map tasks to the reduce tasks.
    \item This shuffling starts as soon as the first map task
      completes.
  \end{itemize}
\end{frame}

\begin{frame}
  \frametitle{Network Transfer Rate}
  \begin{itemize}
    \item The first hump in the graph is for the first batch of
      approximately 1700 reduce tasks (the entire MapReduce was
      assigned about 1700 machines and each machine executes at most
      one reduce task at a time).
    \item Roughly 300 seconds into the computation, some of these
      first batch of reduce tasks finish and we start shuffling data
      for the remaining reduce tasks.
  \end{itemize}
\end{frame}

\begin{frame}
\frametitle{Reducer Writing}
\centerline{\pgfimage[height=0.7\textheight]{reducer-write.png}}
\end{frame}

\begin{frame}
  \frametitle{Reducer Writing Rate}
  \begin{itemize}
    \item There is a delay between the end of the first shuffling
      period and the start of the writing period because the machines
      are busy sorting the intermediate data.
    \item The writes continue at a rate of about 2-4 GB/s for a while
      and all writes finish about 850 seconds into the computation.
  \end{itemize}
\end{frame}

\begin{frame}
  \frametitle{Observation}
  \begin{itemize}
    \item The input rate is higher than the shuffle rate and the
      output rate because of our locality optimization.
    \item The shuffle rate is higher than the output rate because the
      output phase writes two copies of the sorted data (we make two
      replicas of the output for reliability and availability
      reasons).
  \end{itemize}
\end{frame}



\end{document}
